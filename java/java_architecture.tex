\documentclass[a4paper]{article}
 
\usepackage{geometry}
\usepackage[T1]{fontenc}
\usepackage[latin1]{inputenc}

\title{CoMa Architecture}

\date{November 22, 2004}
\author{H. Brandt, P. Kaufels, M. Tiedje, U. Schwarz, O. Wulf}

\begin{document}
\maketitle
\tableofcontents

\section{Implementation Phase 1 (Codename: \textsc{Lethargy})---Core Functionality}
  \subsection{Conference Opening} \label{sec:1:opening}
  Opening the Conference is a separated unit of the implementation of Coma. 
Most of this work will be used only once.
  After the administration started up the environment, it should be possible for the chair
to visit coma's Homepage. There he should enter the conference data. After verifying the data, it is not possible to visit this page again. Then Coma's standard layout comes to light.
\subsection{Authorization/Session Tracking}\label{sec:1:authorization}
Due Authorization is closely related to Session Tracking, it makes sense to see as a separated Part.
Authorization: The invited persons should have a username and a password for logging in.
These passwords and usernames should be generated automatically and send via email.
Session Tracking: it should be possible that several different users are connected at the same time.

	\subsection{Persons: responsible: mti/owu}\label{sec:1:persons}
  In this Phase all Persons have only a small set of actions
  \begin{itemize}
  	\item Person: sign up, update personal data
  	\item Author: check in a paper
  	\item Reviewer: read and rank all paper
  	\item Chair: Conference Opening, invite and set reviewer status, basic viewport to database
  \end{itemize}
 (perhaps we can use inheritance)
  \subsection{Criterion: responsible: ums}\label{sec:1:criterion}
  \begin{itemize}
  \item Grading is essential to COMA.
  \item Nothing outside needs to know about the inside of the
    mechanism, because they can always ask the Criteria themselves to
    \begin{itemize}
    \item emit code for the setup/config process (probably better through
      a factory)
    \item emit code to let a reviewer grade
    \item give eyecandy statistics to the chair
    \end{itemize}
  \item We need some mechanism to implement the abstract data type
    from the start, but not necessarily the fully-customizable
    version.
  \item start with something simple like grades $1$--$6$ and only one
    grade per paper.
  
  \end{itemize}
%
  Proposed interface:
  abstract class ACriterion 
  \begin{description}
  \item [static void init (...params as needed...)] emit code where
    chair can set up the system, initialize factory make().
  \item [static ACriterion make(Paper p, Reviewer r)] return a new
    instance with the fixed setup, passed paper and reviewer.
  \item [void letGrade(User u) throws PermissionException] emit code
    that lets reviewer grade the paper or maybe change the grade. Fail
    if it's not a valid reviewer for the paper.
  \item [{double(?)[] grades}] return the grades. Possibly, the paper
    is responsible for collecting and averaging them. Encapsulating a
    vector space for averagable grades seems excessive.
  \end{description}
   \subsection{Forum: responsible: hbra/pka }\label{sec:1:forum}
 It is not essential to implement the forum in first phase. Depending on final specification we'll see in how far the forum is separated from the main part of implementation(due to implementation second phase).
  \subsection{Layout} \label{sec:1:layout}
  Often in Business developing layout is separated from developing the implementation. Changes in the layout should be easily fulfilled and on this purpose it is useful to have a separation. For example layout standards should be considered.
  \subsection{I18N: responsible: ums}\label{sec:1:i18n}
  \begin{itemize}
  \item We should offer in-principle support for user interfaces in
    multiple languages. (Time considerations aside.)
  \item This facility has to be present from the start, since it is
    very error-prone to replace string constants throughout a finished
    product.
  \item Java already has many features to help this, but (as usual)
    offers much flexibility at the cost of complicated handling.
  \item Primary language is English
  \item Provide wrappers around java.text.MessageFormat et al.
  \item Get language settings from user database.
  \item subject to change as I learn more about Java I18N.
  \end{itemize}
 
  \section{Implementation Phase 2 (Codename: \textsc{Faint})---Extended Functionality}
	\subsection{Persons}\label{sec:2:persons}
	All persons will get more functions and restrictions, e.g. a reviewer can only rate papers, for which he is adept to.
	The Chair gets more overview to the rated and not rated papers and can update the Conference Opening setup.
  \subsection{Criterion}\label{sec:2:criterion}
  In this phase, implement a fuller, more customizable system, such as
  $n$-Tuples of subcriteria, each of which has a title and seperate
  grade ranges, which themselves can have associated explanatory
  strings (such as ``Tension: This paper was: making me fall asleep
  (0); boring (1); interesting (2); making me want to marry the author
  (4)'' Note that while we probably will want numerical values ``under
  the hood'', these need not necessarily be consecutive. (Maybe we can
  get around having something other than arith. mean for reports,
  then.)
 \subsection{Forum}\label{sec:2:forum}
 The forum will be implemented and connected to the authorised persons.
  \section{Implementation Phase 3 (Codename: \textsc{Lassitude})---Bells and Whistles}
  \subsection{Persons }\label{sec:3:persons}
  Write a Cascading Style Sheet for the browser interface.
  \subsection{I18N}
  Have several localizations, via the mechanisms described
  in~\ref{sec:1:i18n}. (Potentially very boring.)
 
\end{document}
% finis
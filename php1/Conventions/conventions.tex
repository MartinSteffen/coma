%% Generelle Dokumenteigenschaften
\documentclass[headexclude,footexclude,12pt,BCOR0pt,DIV15]{scrartcl}
\parindent0.0em

%\documentclass[12pt]{scrartcl}

%% Silbetrennung, Sprache etc...
\usepackage[T1]{fontenc}
\usepackage[german]{babel}
\usepackage[latin1]{inputenc}
\usepackage{ae,aecompl}

%% Links?
%\usepackage[colorlinks=true,urlcolor=webblue,linkcolor=webblue]{hyperref}

%% Listings
\usepackage[scaled]{luximono}
\usepackage{listings}
\usepackage{fancyvrb}
%\begin{Verbatim}[commandchars=\\\{\},codes={\catcode`$=3\catcode`^=7\catcode`_=8}]
% ...
%\end{Verbatim}
\lstset{basicstyle=\footnotesize\ttfamily,numbers=left,stepnumber=5,tabsize=2,stringstyle=\emph,numberstyle=\tiny,numbersep=3pt,frame={tb},breaklines=true}
\newcommand{\code}[2]{\lstinputlisting[title=#1]{#2}}

%% Headings
\usepackage{scrpage2}
\newcommand{\ElveName}{Jan Waller}
\newcommand{\ElveUebung}{1}
\newcommand{\ElveDatum}{\today}
\deftripstyle{ElveStyleUebung}[0pt][0.1pt]{\ElveName}{\ElveUebung}{\ElveDatum}{}{\pagemark}{}
\pagestyle{ElveStyleUebung}

%% PDF, Grafiken
\ifpdfoutput{%
  \usepackage[pdftex]{graphicx}
%  \usepackage[pdftex,plainpages=false]{hyperref}
  \pdfcompresslevel=9
%  \hypersetup{a4paper,pdfborder=0}
  \DeclareGraphicsExtensions{.jpg, .pdf, .mps, .png}
}{%
%  \usepackage[plainpages=false]{hyperref}
  \usepackage{graphicx}
  \DeclareGraphicsExtensions{.eps}
}

%% Sonstiges
\usepackage{enumerate}
\usepackage{amssymb}
\newcommand{\mb}[1]{\ensuremath{\mathbb{#1}}}
\newcounter{an}
\newcommand{\ao}{\par\vspace{3ex}\stepcounter{an}\sffamily\bfseries\large Aufgabe \arabic{an}: \normalfont \normalsize \par}

\endinput

%% %% %% \input{handout}
%% \renewcommand{\ElveInstitut}{Institut f\"{u}r Informatik und Praktische Mathematik}
%% \renewcommand{\ElveProf}{Prof. Dr. W.-P. de Roever}
%% \renewcommand{\ElveSemester}{WS 2004/05}

%% In erster Zeile:
%% \thispagestyle{plain}

\newcommand{\ElveInstitut}{\ }
\newcommand{\ElveProf}{\ }
\newcommand{\ElveSemester}{WS 2004/05}

\AtBeginDocument{
  \begin{minipage}[b]{0.7\textwidth}
    \begin{center}
      \textsc{Christian-Albrechts-Universit\"{a}t zu Kiel} \\
      \ElveInstitut \\
      \ElveProf \\[2ex]
      \ElveName\\
    \end{center}
  \end{minipage}
  \begin{minipage}[b]{0.3\textwidth}
    \begin{center}\includegraphics[height=12ex]{CAU-Siegel}\end{center}
  \end{minipage}

  \hrulefill

  \noindent\ElveSemester\hfill\textbf{\ElveUebung}\hfill\ElveDatum

  \vspace*{4ex}
}

%\deftripstyle{ElveStyleHandout}[0pt][0.1pt]{}{}{}{}{\pagemark}{}

\endinput

%% \renewcommand{\ElveInstitut}{Institut f\"{u}r Informatik und Praktische Mathematik}
%% \renewcommand{\ElveProf}{Prof. Dr. W.-P. de Roever}
%% \renewcommand{\ElveSemester}{WS 2004/05}

%% In erster Zeile:
%% \thispagestyle{plain}

\newcommand{\ElveInstitut}{\ }
\newcommand{\ElveProf}{\ }
\newcommand{\ElveSemester}{WS 2004/05}

\AtBeginDocument{
  \begin{minipage}[b]{0.7\textwidth}
    \begin{center}
      \textsc{Christian-Albrechts-Universit\"{a}t zu Kiel} \\
      \ElveInstitut \\
      \ElveProf \\[2ex]
      \ElveName\\
    \end{center}
  \end{minipage}
  \begin{minipage}[b]{0.3\textwidth}
    \begin{center}\includegraphics[height=12ex]{CAU-Siegel}\end{center}
  \end{minipage}

  \hrulefill

  \noindent\ElveSemester\hfill\textbf{\ElveUebung}\hfill\ElveDatum

  \vspace*{4ex}
}

%\deftripstyle{ElveStyleHandout}[0pt][0.1pt]{}{}{}{}{\pagemark}{}

\endinput

%% \renewcommand{\ElveInstitut}{Institut f\"{u}r Informatik und Praktische Mathematik}
%% \renewcommand{\ElveProf}{Prof. Dr. W.-P. de Roever}
%% \renewcommand{\ElveSemester}{WS 2004/05}

%% In erster Zeile:
%% \thispagestyle{plain}

\newcommand{\ElveInstitut}{\ }
\newcommand{\ElveProf}{\ }
\newcommand{\ElveSemester}{WS 2004/05}

\AtBeginDocument{
  \begin{minipage}[b]{0.7\textwidth}
    \begin{center}
      \textsc{Christian-Albrechts-Universit\"{a}t zu Kiel} \\
      \ElveInstitut \\
      \ElveProf \\[2ex]
      \ElveName\\
    \end{center}
  \end{minipage}
  \begin{minipage}[b]{0.3\textwidth}
    \begin{center}\includegraphics[height=12ex]{CAU-Siegel}\end{center}
  \end{minipage}

  \hrulefill

  \noindent\ElveSemester\hfill\textbf{\ElveUebung}\hfill\ElveDatum

  \vspace*{4ex}
}

%\deftripstyle{ElveStyleHandout}[0pt][0.1pt]{}{}{}{}{\pagemark}{}

\endinput

\renewcommand{\ElveInstitut}{Institut f\"{u}r Informatik und Praktische Mathematik}
\renewcommand{\ElveProf}{Prof. Dr. W.-P. de Roever}
\renewcommand{\ElveSemester}{WS 2004/05}
\renewcommand{\ElveName}{Jan Waller - PHP Gruppe 1}
%\renewcommand{\ElveDatum}{\today}
\renewcommand{\ElveUebung}{P-I-T-M - Guidelines}
\lstset{language=PHP}

\begin{document}
\thispagestyle{plain}
\tableofcontents
%\pagebreak

\section{Coding style}
    This section is based on an article by \emph{Tim Perdue} on \emph{www.phpbuilder.com}.
    I didn't ask for his permission, but most of this is just copy pasted... Some Additions done by me...

    \subsection{Indenting}
        All your code should be properly indented. This is the most fundamental thing you can do to
        improve readability. Even if you don't comment your code, indenting will be a big help to anyone
        who has to read your code after you.

        \begin{lstlisting}[stepnumber=0,frame={}]
while ($x < $z) {
  if ($a == 1) {
    echo 'A was equal to 1';
  } else {
    if ($b == 2) {
      //do something
    } else {
      //do something else
    }
  }
}
        \end{lstlisting}

        Use only spaces for indenting code, never tabs.  Tab display width is not standardized enough,
        and anyway it's easier to manually adjust indentation that uses spaces.

        As a standard I propose a width of 2.

        Stay within 80 columns, the width of a minimal standard display window.

    \subsection{Braces}
        If you use conditional expressions (\texttt{IF} statements) without braces, not only is it less readable,
        but bugs can also be introduced when someone modifies your code.
        \begin{description}
        \item{Bad Example:}
\begin{lstlisting}[stepnumber=0,frame={}]
if ($a == 1) echo 'A was equal to 1';
\end{lstlisting}

        That's pretty much illegible. It may work for you, but the person following after you won't appreciate it at all.

        \item{Less Bad Example:}
\begin{lstlisting}[stepnumber=0,frame={}]
if ($a == 1)
  echo 'A was equal to 1';
\end{lstlisting}

        Now that's at least readable, but it's still not maintainable. What if I want an additional action to occur when
        \texttt{\$a==1}? I need to add braces, and if I forget to, I'll have a bug in my code.

        \item{Correct:}
\begin{lstlisting}[stepnumber=0,frame={}]
if (($a == 1) && ($b==2)) {
  echo 'A was equal to 1';
  //easily add more code
} elseif (($a == 1) && ($b==3)) {
  //do something
}
\end{lstlisting}

            Notice the space after the \texttt{if} and \texttt{elseif} - this helps distinguish conditionals from function calls.
        \end{description}

    \subsection{Function Calls}
        Functions should be called with no space between between the function name and the parentheses.
        Spaces between params and operators are encouraged.

\begin{lstlisting}[stepnumber=0,frame={}]
$var = myFunction($x, $y);
\end{lstlisting}

    \subsection{Functions}
        Function calls are braced differently than conditional expressions. This is a case where I'll have
        to change my personal coding style in order to be kosher, but I guess it needs to be done.

\begin{lstlisting}[stepnumber=0,frame={}]
function myFunction($var1, $var2 = '')
{
  //indent all code inside here
  return $result;
}
\end{lstlisting}

        Notice again there is no space between the function name and the parens, and that the params are nicely
        spaced. All your code inside the function will be at least 4 spaces indented.

        Another important principle when coding functions is that they should always return instead of print directly.
        Remember, if you print directly inside your function call, whomever calls your function cannot capture its
        output using a variable.

    \subsection{Comments}
        Borrowing - almost in its entirety - from the \texttt{JavaDoc} spec, I strongly encourages the use of \texttt{PHPDoc} comment style.
        JavaDoc is rather clever because you format your code comments in such a manner that they can be parsed by a
        doc-generating tool, essentially creating \emph{self commenting} code. You can then view the resulting docs using a web browser.

\begin{lstlisting}[stepnumber=0,frame={}]
/**
 *  short description of function
 *
 *  Optional more detailed description.
 *
 *  @param $paramName - type - brief purpose
 *  @param ...
 *  ...
 *  @return type and description
 */
\end{lstlisting}


    \subsection{Including Code}
        PHP4 introduced a pretty useful new feature: \texttt{include\_once()}. I've seen a lot of questions on the mailing
        lists and discussion boards caused by people including the same files in multiple places. This can cause conflicts
        when the included files include functions, which can be defined only once per script.

        The simple solution is to replace all your \texttt{include()} and \texttt{require()} calls with corresponding
        \texttt{include\_once()} and \texttt{require\_once()}. Both use the same file list, so a file included with
        \texttt{require\_once()} will not be later included with an \texttt{include\_once()} call. (technically,
        \texttt{require\_once} and \texttt{include\_once} are \emph{operators}, not \emph{functions} so parens are not necessary).

        I never use \texttt{include}, or any of these \texttt{*\_once} functions. IMHO, well-designed applications include
        files in one place that is easy to maintain and keep track of. Others will disagree and say that each included file
        should \texttt{require\_once()} every file that it depends on. This seems like extra overhead and maintenance headaches
        to me personally. Either way, the days of conflicting includes are probably over.

    \subsection{PHP Tags}
        You should always use the full \texttt{<?php ?>} tags to enclose your code, not the abbreviated \texttt{<? ?>} tags,
        which could conflict with XML or other languages. ASP-style tags, while supported, are messy and discouraged.

    \subsection{Strings}
        In PHP of course, double quotes (") are parsed, but single quotes (') are not. That means PHP does extra work
        (magic really) on double-quoted strings that it doesn't do on single-quoted strings. So there is a (subtle)
        performance difference between the two, especially in code that is iterated or called a large number of times.
        \begin{description}
            \item{Best:}
\begin{lstlisting}[stepnumber=0,frame={}]
$associative_array['name'];
$var='My String';
$var2='Very... long... string... ' . $var . ' ...more string... ';
$sql="INSERT INTO mytable (field) VALUES ('$var')";
\end{lstlisting}
            \item{Acceptable:}
\begin{lstlisting}[stepnumber=0,frame={}]
$associative_array["name_$x"];
$var="My String $a";
\end{lstlisting}
        \end{description}

        The first example does not include any \texttt{vars} that need to be appended or parsed into the strings,
        so single quotes (') are definitely the best way to go. The second example includes very short strings and
        has variables that need to be parsed into the strings, so it is acceptable to use double quotes. If the strings
        were long, it would be best to use the append (.) operator.

        For consistency's sake, I always use single quotes around all my strings, except SQL statements, which almost
        always contain apostrophes and variables that must be parsed into the strings.

\section{Writing log messages}
    This section is again almost entirely copied (this time my source
    is the \texttt{papidSVN} Hacking Guide...\\
    \\
    Certain guidelines should be adhered to when writing log messages:
    \begin{itemize}
    \item Make a log message for every change.  The value of the log becomes
        much less if developers cannot rely on its completeness.  Even if
        you've only changed comments, write a log that says \emph{Doc fix.} or
        something.

    \item Use full sentences, not sentence fragments.  Fragments are more often
        ambiguous, and it takes only a few more seconds to write out what you
        mean.  Fragments like \emph{Doc fix}, \emph{New file}, or \emph{New function} are
        acceptable because they are standard idioms, and all further details
        should appear in the source code.

    \item The log message should name every affected function, variable, macro,
        makefile target, grammar rule, etc, including the names of symbols
        that are being removed in this commit.  This helps people searching
        through the logs later.  Don't hide names in wildcards, because the
        globbed portion may be what someone searches for later.  For example,
        this is bad:
\begin{Verbatim}
   * twirl.cpp
     (twirling_baton_*): Removed these obsolete structures.
     (handle_parser_warning): Pass data directly to callees, instead
     of storing in twirling_baton_*.

   * twirl.h: Fix indentation.
\end{Verbatim}
        Later on, when someone is trying to figure out what happened to
        `twirling\_baton\_fast', they may not find it if they just search for
        "\_fast".  A better entry would be:
\begin{Verbatim}
   * twirl.cpp
     (twirling_baton_fast, twirling_baton_slow): Removed these
     obsolete structures.
     (handle_parser_warning): Pass data directly to callees, instead
     of storing in twirling_baton_*.

   * twirl.h: Fix indentation.
\end{Verbatim}

        The wildcard is okay in the description for `handle\_parser\_warning',
        but only because the two structures were mentioned by full name
        elsewhere in the log entry.

        Note how each file gets its own entry, and the changes within a file
        are grouped by symbol, with the symbols are listed in parentheses
        followed by a colon, followed by text describing the change.  Please
        adhere to this format -- not only does consistency aid readability, it
        also allows software to colorize log entries automatically.

    \item If your change is related to a specific issue in the issue tracker,
        then include a string like "issue \#N" in the log message.  For
        example, if a patch resolves issue 1729, then the log message might
        be:
\begin{Verbatim}
   Fix issue #1729:

   * get_editor.cpp
     (frobnicate_file): Check that file exists first.
\end{Verbatim}

    \item For large changes or change groups, group the log entry into
        paragraphs separated by blank lines.  Each paragraph should be a set
        of changes that accomplishes a single goal, and each group should
        start with a sentence or two summarizing the change.  Truly
        independent changes should be made in separate commits, of course.

    \item One should never need the log entries to understand the current code.
        If you find yourself writing a significant explanation in the log, you
        should consider carefully whether your text doesn't actually belong in
        a comment, alongside the code it explains.  Here's an example of doing
        it right:
\begin{Verbatim}
   (consume_count): If `count' is unreasonable, return 0 and don't
   advance input pointer.
\end{Verbatim}
        And then, in `consume\_count' in `cplus-dem.cpp':
\begin{Verbatim}
   while (isdigit ((unsigned char)**type))
   {
     count *= 10;
     count += **type - '0';
     /* A sanity check.  Otherwise a symbol like
       `_Utf390_1__1_9223372036854775807__9223372036854775'
       can cause this function to return a negative value.
       In this case we just consume until the end of the string.  */
     if (count > strlen (*type))
     {
       *type = save;
       return 0;
     }
\end{Verbatim}
        This is why a new function, for example, needs only a log entry saying
        \emph{New Function} --- all the details should be in the source.

    \item There are some common-sense exceptions to the need to name everything
        that was changed:
        \begin{itemize}
        \item If you have made a change which requires trivial changes
          throughout the rest of the program (e.g., renaming a variable),
          you needn't name all the functions affected, you can just say
          \emph{All callers changed}.

        \item If you have rewritten a file completely, the reader understands
          that everything in it has changed, so your log entry may simply
          give the file name, and say "Rewritten".

        \item If your change was only to one file, or was the same change to
          multiple files, then there's no need to list their paths in the
          log message (because "svn log" can show the changed paths for
          that revision anyway).  Only when you need to describe how the
          change affected different areas in different ways is it
          necessary to organize the log message by paths and symbols, as
          in the examples above.
        \end{itemize}

    \item In general, there is a tension between making entries easy to find by
        searching for identifiers, and wasting time or producing unreadable
        entries by being exhaustive.  Use your best judgment --- and be
        considerate of your fellow developers.  (Also, run "svn log" to see
        how others have been writing their log entries.)
    \end{itemize}

\section{SVN specific stuff}
    \begin{itemize}
        \item If you add any files to the project that might be opened on more than one operating system then give the file
            the native eol-style property.  This ensures that all files downloaded via svn arrive with the line endings
            native to that OS.  Here is how you add the property:\\
            \texttt{\% svn propset eol-style native new\_file.php}

        \item If you add images to the project make sure you apply the correct  mime-type property.\\
            Example for a XPM:\\
            \texttt{\% svn propset svn:mime-type image/x-xpm}
    \end{itemize}


\end{document}

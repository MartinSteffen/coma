\documentclass[a4paper]{article}
\usepackage[dvips]{graphics}
\pagenumbering{arabic}
\title{Tools and Technologies: COMA (conference management system)}
\author{Alexander Derenbach}

\begin{document}
	\maketitle
	\section{Introduction}
	This is a description of the tools that should be used to implement COMA. 
	Concept and specification are covered by other papers.
	\section{The Architecture}
	COMA should be implemented as a client/server architecture and be accessible over the Internet. On account of this it is consequent to use an available server/client architecture such as webserver/webbrowser. The following section will discuss some of the technologies and software products in this field.
	\section{The Tools}
	\subsection{The Webserver}
	The available webservers can be classified into 'free' and 'commercial' products.\\
	Commercial, that is you have to pay for them, are:
	\begin{itemize}		
		\item Netscape Enterprise Server (commercial)
		\item IPlanet (commercial)
		\item MS IIS (commercial, only MS Server / Windows Platform)	
	\end{itemize}	
	The free webservers are:
	\begin{itemize}	
		\item Apache + modules (free)
		\item Roxen (free)
	\end{itemize}
	Out of these we choose Apache as our target server for these reasons:
	\begin{itemize}	
		\item It's free and under GPL (GNU Public License).
		\item It's a good and fast webserver.
		\item There exists already some know-how within the team.
		\item It supports almost all common concepts by modules or add-ons (Java, PHP, Perl, etc.)
	\end{itemize}
	\subsection{The DBMS (Database Management System)}
	Available DBMS are:
	\begin{itemize}
		\item MySQL 4.1
		\item PostgreSQL
		\item Orcale 
		\item DB2
		\item Informix
	\end{itemize} 
	Out of these we choose MySQL 4.1 as our target DBMS for these reasons:
	\begin{itemize}
		\item It's free and under GPL (GNU Public License).
		\item It's available on most platforms.
		\item There are drivers available for most languages.
	\end{itemize}
Note:  It has to be MySQL version 4.1 since older versions don't support SQL-Subquery syntax
	\subsection{The Language}
	Available Languages / Technologies:
	\begin{itemize}
		\item Java
		Pros:
		\begin{itemize}
			\item modern object orientated language
			\item platform independent
			\item clear seperation of moduls possible
			\item easy to automate testing the core  
		\end{itemize} 
		Cons:
		\begin{itemize}
			\item slower than some other technologies
			\item partitial heavyweight
		\end{itemize} 
		
		\item PHP

		Pros:
		\begin{itemize}
			\item relativ fast
			\item platform independent
			\item developed for display dynamic data on webclients
		
		\end{itemize} 
		Cons:
		\begin{itemize}
			\item complex seperation of modules
			\item less existing knowhow
			\item tesing more complex
		\end{itemize} 
		\item Perl

		Pros:
		\begin{itemize}
			\item platform independent
		\end{itemize} 
		Cons:
		\begin{itemize}
			\item complex syntax
			\item antiqued for webservices 
		\end{itemize} 
		\item ASP

		Pros:
		\begin{itemize}
			\item relativ fast
			\item developed for display dynamic data on webclients
		\end{itemize} 
		Cons:
		\begin{itemize}
			\item not not offical supportet for all platforms. Indeed a module for apache exists but not shure if it is in a stable business suitable release. 
		\end{itemize}
	\end{itemize} 
	Out of these we choose Java as our target language for these reasons:
	\begin{itemize}
		\item There exists already some know-how within the team.
		\item clear seperation of components (core, data, view)
		\item the language concept suits to teamwork concepts (easy to
	seperate modules)
		\item possible to automize testing with JUnit
	\end{itemize}	    
	\section{Realisation in Java.}
	
	Java Technologies:
	\begin{itemize}		
		\item Java 2 
		\item Java-Servlet 
		\item EnterpriseJavaBeans EJB
		\item JSP 
		\item JSP Tag Libraries 
		\item MySQL� Connector/J
		\item Jakarta Tomcat 
		\item JBoss Application Server
	\end{itemize}
Java Servlets, JSP and EJB are used for a clear seperation of functionality and view. 
\newpage
\section{Links}
\begin{tabbing}
	namefillUPwit:\=link1fillUPwi:\=link2fillUPwit:\kill
	Apache \>\>  http://www.apache.org\\
	MySQL \>\>   http://www.mysql.org\\
	J2SE \> \>   http://java.sun.com/j2se/index.jsp\\
	J2EE \>\>	   http://java.sun.com/j2ee/index.jsp\\
	EJB \>\>	   http://java.sun.com/products/ejb/\\
	JSP \>\>	   http://java.sun.com/products/jsp/index.jsp\\
	Servlets \>\> http://java.sun.com/products/servlet/index.jsp\\
	JDBC:mysql \>\> http://www.mysql.com/products/connector/j/\\
	Jakarta Tomcat \>\> http://jakarta.apache.org/tomcat/index.html\\
	Jboss App. Server \>\> http://www.jboss.org/\\
	\end{tabbing}
\end{document}
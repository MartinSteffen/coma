\documentclass{beamer}


\setbeamertemplate{background canvas}[vertical shading][bottom=white,top=structure.fg!25]
% Oder was auch immer.

\usetheme{Warsaw}
%\setbeamertemplate{headline}{}
%\setbeamertemplate{footline}{}
\setbeamersize{text margin left=0.5cm}
  
\usepackage[german]{babel}
% Oder was auch immer

\usepackage[latin1]{inputenc}
% Oder was auch immer
%\usepackage{times}
%\usepackage[T1]{fontenc}
% Oder was auch immer. Zu beachten ist, das Font und Encoding passen
% m�ssen. Falls T1 nicht funktioniert, kann man versuchen, die Zeile
% mit fontenc zu l�schen.
\title{Gruppe Tools 2}
\author{H.Brandt, P.Kaufels, M.Tiedje}

\begin{document}

\begin{frame}
	\frametitle{	Welche Sprachen sollen verwendet werden? \textsc{Java, Sql92}}
	\begin{itemize}
  	\item \textsc{Java (Servlets/JSP)}
    \begin{itemize}
    	\item im Kurs am verbreitetsten
    	\item nur eine Sprache $\Rightarrow$ einfachere Integration
    	\item echte Objektorientierung 
    	\item robust, effizient, einfach
    	\item gute Fehleranalyse/-behandlung: viele Fehler treten schon zur Compilezeit auf
 			\item gute Wartbarkeit, sehr gute Dokumentation
    	\item Datenaustausch und Kommunikation:\\- Servlets k�nnen sich Daten teilen\\- XML Verarbeitung mit SAX/DOM
    	\item Klassenbibliotheken f�r Session-Trecking, Formulare oder Cookies
    	\item Datenbankanbindung via JDBC
    \end{itemize}
    \item \textsc{Sql92}
    \begin{itemize}
    	\item ist der Industriestandard f�r Datenbanken, damit interoperabel 
    \end{itemize}
	\end{itemize}
\end{frame}

\begin{frame}
	\frametitle{ Welche Software soll verwendet werden? \textsc{Tomcat, MySql}}
  \begin{itemize}
  	\item \textsc{Tomcat:}
    \begin{itemize}
    	\item	offizielle Referenzimplementierung von Servlets/JSP
    	\item frei, weit verbreitet
    	\item hervorragend dokumentiert
    \end{itemize}
    \item \textsc{Mysql:}
    \begin{itemize} 
    	\item f�r Entwickler frei, weit verbreitet
    	\item ist ANSI SQL92 kompatibel
    	\item "`schlank"'
    	\item viele freie Tools verf�gbar z.B. phpMyAdmin
   	\end{itemize}
   	\item s�mtliche Software kann in einer lokalen Umgebung (z.B.: zu Hause) installiert werden
 	\end{itemize}
 	
\end{frame}	

\begin{frame}
	\frametitle{Versionen,Quellen und Doku}
  \begin{itemize} 
  	\item \textsc{Java: J2SE 5.0}
  	\begin{itemize} 
  		\item generische Programmierung
  		\item f�r Tomcat 5.5 erforderlich
  		\item http://java.sun.com/j2se/1.5.0/download.jsp
  		\item Online Buch: http://www.galileocomputing.de/openbook/javainsel4/
  	\end{itemize}
  	\item \textsc{Tomcat: 5.5}
  	\begin{itemize} 
  		\item http://mirrorspace.org/apache/jakarta/tomcat-5/v5.5.4/
  		\item Doku: http://jakarta.apache.org/tomcat/tomcat-5.5-doc/index.html
  	\end{itemize}	
  		\item \textsc{MySql 4.1}
  	\begin{itemize} 
  		\item http://dev.mysql.com/downloads/mysql/4.1.html
  		\item Doku: http://dev.mysql.com/doc/mysql/en/index.html
  	\end{itemize}
  	\item \textsc{Xhtml 1.1}
  	\begin{itemize} 
  	\item vollst�ndig XML konform, modular
  	\item Trennung von Inhalt und Layout mit CSS
  	\item Doku:\\ - http://www.w3.org/TR/xhtml11/ \\ - http://de.selfhtml.org/
  	\end{itemize}
	\end{itemize}
\end{frame}


\end{document}



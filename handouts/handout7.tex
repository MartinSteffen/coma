\newif\ifweb\webfalse

\documentclass[11pt,handout,nologo]{handout}

\dozent{Marcel Kyas, Gunnar Schaefer Martin Steffen}
\coursename{P-I-T-M: Coma}
\semester{WS 2004/05}
\ausgabetermin{31. Januar 2005}
\handouttitle{not needed, only for freeform}
\nummer{7}
\thema{E-O-S: Judgment day}


\newcommand{\Coma}{\textsl{Coma}}
\ifweb
\usepackage{hevea}
\fi
\usepackage{hyperref}
\usepackage{psfrag}

\usepackage{graphicx}
\graphicspath{{../figures/}} %% cool!


%\uebungfalse
%\handouttrue

\begin{document}

\thispagestyle{empty}





\begin{abstract}
  This handout clarifies and lays down the rules and schedule for the end
  of semester presentation, which we so far announced and discussed only
  informally.
\end{abstract}



\subsection*{Where, what, when, who \ldots}

The goal of the \emph{E-O-S presentation} for you is to present the results
of your efforts to us and your student colleagues. The presentation should
contain a ``talk'' part and a ``demo'' part (not necessarily sequentially
separated). It takes place
\begin{quotation}
  8.2.05, 8o'clock -- 12 o'clock, in {\"U}2, our normal meeting room.
\end{quotation}

We propose, that every group, including the testers as global group, gets a
\emph{slice of 30 minutes,} calculated as if there were no questions.
Probably, there will be some questions or delays (demo failure \ldots) for
which allot estimated 10 minutes. Apart from that, the individual tester
should fill perhaps 10 minutes within the slot of the respective group, to
stress his contribution and integration within his group, the errors
found, repaired etc.

The (proposed) schedule is shown on in Table~\ref{tab:schedule}



\begin{table}[htbp]
  \centering
  \begin{tabular}[t]{l|llll}
    start &  who & core &  test & questions
    \\\hline
    8:15 & tests & 30 & & 10
    \\
    8:45 & PHP 1 & 30 & 10 &10
    \\
    9:35 & \multicolumn{4}{c}{break}
    \\
    9:45 & PHP2 & 30 &10 &10
    \\
    10:35 & Java & 30 & 10 & 10
    \\
    11:25 & org & rest
  \end{tabular}
  \caption{Schedule}
  \label{tab:schedule}
\end{table}

Here a few hints and ideas what you could address in your presentation and
point that you should be aware of.

\begin{description}
\item[Demo:] the machine, on which the server runs, is good old
  \emph{snert}. Your demo should contain 2 parts
  \begin{enumerate}
  \item \emph{installation-demo:} for the installation, this should be
    performed by an outsider, on a \emph{fresh} account on snert. So
    provide some ``installation description'' (preferably fool proof, and
    not too complex), that you hand out to the ``fool'' who installs your
    tool on the new account. You may assume that the person can read \ldots
  \item \emph{tool-in-action:} ideally, your tool of course continues
    working after installation. However, to avoid delays (failure of
    installation and especially because after installation the data base is
    empty) you may switch to a running and already filled server, where you
    can demonstrate features of your tool. This part can be done by yourself
    (but the audience may require thing like \emph{``What happens if you
      now klick down there''})
  \item \emph{tool-under-stress:} depending on how you plan to integrate
    the test-part of your part, also running tests could be part of the
    demo.
  \end{enumerate}
\item[presentation:] you are required to present core ideas about your
  tool, your design, your group etc. Points you can address in that part are
  \begin{itemize}
  \item (short!) intro (no extensive blabla, no ``commercials'' \ldots, we
    do a moderately technical show.)
  \item structure of the tool: architecture, responsibilities, features,
    highlights, what's missing,
  \item about the \emph{development process}:
    \begin{itemize}
    \item \emph{locally} in your group: how did you cooperate/coordinate?
      How much of your original plan was done? If something is missing,
      what went wrong? Which decisions have turned our good, which bad.
    \end{itemize}
  \item \emph{globally:} similar things about the global set-up of the
    course. What (if we would do it again) would you prefer to be done
    differently? And why?
  \end{itemize}
\end{description}

We expect that everyone is present at the demos.\footnote{With certain
  justified exceptions, for instance ``Klausur''. Ask for this exceptions,
  as they must be granted by us. Just not showing up and the rest of the
  groups tells us at showtime, is not enough.}  It's recommended, that you
split duties during the presentation, it makes a bad impression, of one guy
takes all of the burden and the rest sits in the background, nodding at
the right moments; especially if you are one of the less visible
participant, take the chance and show yourself. Even if the presentation
and the sharing of duties in the show is in your hand, be prepared that not
just the front man is answered question, especially if you are silent all
during semester. Of course, you need not know all details of all modules,
but we expect a general overview over the whole thing, it's structure etc.
from everyone.

You are of course encouraged to ask also questions about the other tools,
if interested.




\subsection*{Technical support:}
\begin{description}
\item[Client machines:] there will be two laptops, with Linux running. One
  will have net access. This you can use to logon snert or to surf at
  snert. The machines have \textsl{acroread} running, so if wanted, one can
  be used for pdf slides. In case you want us to store slides on that
  machine, we need the file till \emph{Monday evening} before the demo.
\item[fresh accounts:] we will provide you with an fresh account on snert
\item[own laptops:] you can of course use your own laptop. Probably, you
  won't be able to hook-on via ethernet-cable/DHPC, however. Whether WLAN
  works there, we don't know.
\item[beamers:] there will be 2 beamers. Our laptops do synchronize with
  the beamer (we tested it).
\end{description}


If you have further requirements: \textbf{announce them in time to us!}
Advice: prepare the demo! A demo which runs for the first time during show
time, is bound to be surprising for the demonstrator \ldots Needless to
say: Trying to run the demo means: trying under the \emph{actual
  circumstance \ldots.}





\end{document}

%%%%%%%%%%%%%%%%%%%%%%%%%%%%%%%%%%%%%%%%%%%%%%%%%%%%%%%%%%%%%
%% $Id: handout3.tex 23 2004-11-11 07:22:29Z ms $ 
%%%%%%%%%%%%%%%%%%%%%%%%%%%%%%%%%%%%%%%%%%%%%%%%%%%%%%%%%%%%%

%%% Local Variables: 
%%% mode: latex
%%% TeX-master: t
%%% End: 

%%% Local Variables: 
%%% mode: latex
%%% TeX-master: t
%%% End: 

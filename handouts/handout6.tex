

\newif\ifweb\webfalse

\documentclass[11pt,handout,nologo]{handout}

\dozent{Marcel Kyas, Gunnar Schaefer Martin Steffen}
\coursename{P-I-T-M: Coma}
\semester{WS 2004/05}
\ausgabetermin{30.~November 2004}
\handouttitle{not needed, only for freeform}
\nummer{6}
\thema{Some remarks on the way we communicate}


\newcommand{\Coma}{\textsl{Coma}}
\ifweb
\usepackage{hevea}
\fi
\usepackage{hyperref}
\usepackage{psfrag}

\usepackage{graphicx}
\graphicspath{{../figures/}} %% cool!


%\uebungfalse
%\handouttrue

\begin{document}

\thispagestyle{empty}





\begin{abstract}
  The handout is an ad-hoc document collecting observations and remarks (as
  far as we see it) concerning the current state of communication problems.
  The handout basically condense and makes explicit the stuff that has been
  discussed/presented during the meeting. So the document probably does not
  contain eternal truths, but just points that currently are of concern and
  which we should keep in mind so that the project(s) and discussions do
  not get out of hand. Compared to the distributed version, I leave out the
  \emph{technical} points concerning the data model and concentrate on the
  ``communication'' part.
\end{abstract}


\iffalse
\subsection*{Open points}

I list current open question, that I've seen on the bulletin board.

\begin{itemize}
\item 
  \begin{enumerate}
  \item \texttt{Conference.desciption}: it's in the textual code, but not in
    the SQL-code. Recommendation: remove in the SQL thing
  \item \textbf{Person.conference:} varchar \texttt{29} to \texttt{20:}
    \texttt{20} is ok, do this in \texttt{SQL}
  \item \textbf{roles:} Int or Bit Vector: not so crucial, but agree
  \item \textbf{paper.id} Bigint: in the textual form: it's int now, in SQL it's
    bigint. agree
  \item \textbf{CoAuthorOf:} field email not necessary (or at least is allowed
    to be zero): Currently it's non-null. Name would suffice. In the UML
    diagram, a set of strings was proposed
  \item \textbf{Forgotten: a reviewer reject to review a paper:} add an assoc
    \texttt{WantsNotReview}
  \item \textbf{must\_not\_review}: This was an association, which was proposed
    in the UML diagram, associating \texttt{Person} and \texttt{Paper}
  \item \textbf{where are grades?:} 
  \item \textbf{Value names}:
  \item \textbf{Forum title:}
  \item \textbf{Forum description:}
  \item \textbf{How to assoc. a forum to a paper?}
  \item \textbf{ParticipatesInForum} to \texttt{ParticipantsInForum}: not yet
    done in SQL
  \end{enumerate}
\item 
  \begin{enumerate}
  \item \textbf{HasThread:} that has been removed
  \item \textbf{User:} only per conference? In other words, if a user wants to
  participate in various conferences, he has to login separately? Answer is:
  yes, that's intended. And I agree with Sandro
\item \textbf{In Person:} reference to rights,id. I could not find that
\item \texttt{redundant references:}
  \begin{itemize}
  \item  from \texttt{Person} to  \texttt{Conference} 
  \item from \texttt{User} to \texttt{Rights}
  \item from \texttt{Rights} to \texttt{Roles}
  \item from \texttt{Roles} to \texttt{Conference}
  \end{itemize}
\item \textbf{redundant reference:} 
  \begin{itemize}
  \item  in \texttt{isCoAuthor} to \texttt{Paper}
    and to \texttt{Conference}. In principle, I agree. Depends however, how
    complex the coauthor is modelled.
  \item \texttt{IsAboutTopic} connects \texttt{Paper} and \texttt{Conference}
    via \texttt{Topic}, but there's also a direct connection from \texttt{Paper}
    to \texttt{Conference}
  \item \texttt{ReviewReport} to \texttt{Conference}
  \end{itemize}
\item Remark: I agree with the comment of Sandro: Paper and Topic should be
  connected to the conference.
\end{enumerate}
\item Further stuff:
  \begin{itemize}
  \item Mohamed is the only one to adapt SQL
  \item Sandro defends his proposition concerning roles., ivanilas  has a more
  complex proposal
\item Discussion about \textbf{IDs}. The arguments on the table are
  \begin{itemize}
  \item \emph{efficiency/perfomance:} I think it's not so important.
  \item \emph{changeability:} what if a user changes his email. I think that's
    a valid valid
  \item \emph{uniqueness:} what if there are different writings of an email
  \item \textrm{readability/understandability:} I'd say: not so relevant
  \item a real problem is indeed: one user in different conferences, but one
    can solve it via \textbf{Ids} or without
  \item adaptability: do we want to find later new roles? My remark: now, we
    have problems enough. I agree with Marcel, we should not design for
    adaptability in this respect and in this way. Besides that, using integers
    because there's `` still so much space left in \texttt{Int}, which we can
    later use, perhaps, if we have something forgotten'', it sounds not
    convincing, it sounds not even like \emph{adaptability,} more like leaving
    the door open for a future hack, if we need one :-) There's still the
    disagreement about the extra table for the roles. We must take a decision.
  \end{itemize}
\item\textbf{Exclude paper:} missing
  \end{itemize}
\fi
\begin{itemize}
\item \textbf{Request of ``Stop the discussion'':} (by Tom)
  \begin{itemize}
  \item The discussion can indeed turn into a problem. As yesterday already
    indicated in the general email: underlying may be (amongst other
    things) a communication problem. One of the core messages I do support
    from what Tom (others too) remarked: we should come to an end.
    Furthermore, the discussions on the board start to get into an exchange
    of arguments for sake of the argument (``and here is yet another reason
    why I'm right and you are wrong \ldots'':). We have to be careful here
    and come to a conclusion, in particular in cases where it does not
    matter what we decide (bitvector vs. integer, the arguments are on the
    table so we go ahead.)
  \item The second underlying message (``If we only had taken the carefully
    and painfully worked out good specification, then we would not had
    those problems'') I do not completely support (I pointed that out on
    the board). It's true that the specs had been worked out well, but we
    are currently somehow deep in ``Varchar(29)'' (a typo) or ``BigInt''
    vs. ``Int'' and all this kind of stuff, and also inconsistency remarks.
    
    I would rather guess, that both of the two delivered specs would have
    drawn equal flak as well, once it came to the details or it came to
    realizing them really. We might have saved, I agree, a week or so, when
    just taking one of the specs and pour them into SQL (because the
    discussion would have started just one week earlier) but some discussion
    would have evolved.  
    
    The counter argument, ``perhaps the previous specs had been some open
    points too, but at least \emph{we} had to implement it'' does not
    really hold in my eyes, because we \emph{all} have to implement it, not
    just each one his own spec. Life would indeed be easier if each group
    could implement whatever it chooses (spec 1, or spec 2, or some
    additions/subset), but that's not the idea.
  \item Related to the above concern (``general development, apart from
    technical details'') I further agree with Sandro (if that was his point):
    No unannouced critical changes which influences someone else. This holds
    even for \emph{healthy} changes.\footnote{Unless one really trusts each
      other.} 
  \end{itemize}
\item And again related to the above: \textbf{the means of communication.}
    \begin{itemize}
    \item I can't speak for the rest of the audience, but it could be that
      this BB is not the fastest means of communication, in particular for
      \emph{agreement.} This is not to propose email instead, but
      \emph{sometimes} to come together and solve trivial technical things
      and that's it. What I mean is that for some issue the net time to
      follow long discussions, browing, reading through quotes etc, is
      longer than the net time to sit together (or make an appointment with
      (me and/or Marcel and/or Gunnar), drop by and hammer it out. Also
      face-to-face sometimes makes compromise easier, and that's important
      as well.
    \item This was raised (by Gunnar) commenting on ``agreements in the
      forum''. I agree with him in this point. Technically, since we (so far)
      said in the meeting, that the forum does \emph{not} constitute an
      obligatory means of communication, there can't be \emph{obligatory}
      agreements concerning topics of general interest on the forum.
      
      A different thing would be that group \emph{X} decides that everyone
      in this group \emph{must} read the relevant group and thus the people
      are forced to use the board. In this point, I agree with
      \emph{Gunnar}. \texttt{Scherbengestalt} mentioned: ``I don't see
      what's going on'' (that's perhaps an indication of internal
      communication problems) and ``there are too many communication
      channels''. For the last point, that's difficult to change. The only
      thing that one can change is to reduce the \emph{obligatory} channels
      (if someone likes to send someone else an SMS about the project, it's
      difficult to do something about).
    \item Here's the means of communication that we have:
      \begin{itemize}
      \item \textbf{snv logs:} I do not recommend this as means of
        communication, unless we decide this, but I think it does not work,
        and we have enough other channels. Especially now, where somehow
        lot's of stuff seems to get checked in, we would be swamped by
        reading meaninless logs. As a matter of fact, many of the logs (by
        me and also by others) are plain empty.
      \item \textbf{after-the-fact Readmes:} I'd recommend that Readmes are
        short and pretty \emph{stable} (up-to the point of course, that
        things they describe change) but not used as a means of
        communication of changes. In this way, it would be more a
        \textbf{log-file.}  Especially, it can turn out contraproductive,
        to change someone else's work and note it down in the Readme (I
        don't know whether we had instances of that already).
      \item Also, in the medium run a bad means (or a sign of bad)
        communication: \textbf{``hand-versioned''} files like
        \texttt{file-x} checked in. At least the alternatives \emph{must}
        as soon as possible been resolved.
        
        In some sense, it's the polite way of avoiding the previous point
        (just overwrite or undo someone else's stuff and the other one has
        to swallow it or re-redo it). In the medium run, it's equally bad,
        because we will loose overview. For instance, currently we have
        \emph{3 versions} of the data-model in textual form lying around
        (two in the repos + one posted on the board accessible via some
        url) and 1 in SQL format. All versions are different. Version 2
        seems to be the evolution of version 1 (so 1 is obsolete), one can
        guess that. Apart from that, however,\footnote{and besides the fact
          that we have to come to a conclusion, which is important \ldots.}
        there creep in slight inconsistencies, for instance things that had
        been resolved in version 2, suddenly re-appear in version 3, which
        makes a proposal to solve one point in version 2, but
        \emph{reentroduces} stuff that had been wrong in version 1.
        
        The danger therefore is, that one looses overview (and we are
        talking here about only \emph{one} (important) file.  This means,
        it is (also) a problem of communication, because one forces the
        others to go through the code again, compare it, see new additions,
        see that old stuff has not been taken into account. Basically, it's
        the same situation as in the previous point, only more ``polite''
        at the surface.
      \item \textbf{Bugzilla} has not yet proved how helpful it will be,
        but I guess it is. By not yet proven I mean: so far it has not been
        put to the (real) test. We have 18 reported bug, and one can
        imagine to keep track of 18 in one's brain alone; only the future
        will tell how many bugs are reported \ldots
      \item \textbf{email:} In my eyes, email so far is not a problem, at
        least not the \emph{official emails} via \texttt{swprakt+coma},
        basically there's one or 2 emails per week. How groups use email
        internally, we don't exactly know.
      \item\textbf{Face-to-face:} probably, in this phase, we need to use
        this more \ldots Especially, we would like to recommend also to
        make separate appointments (with each group) apart from the general
        meetings. This will cost (also us) time, but is probably effective
        in the current stage.
      \item \textbf{Web-page:} we maintain the web-page as server for
        information (keeping the handouts, specs,, the announcemetn emails
        etc) I think, that's important.
      \end{itemize}
      The remarks about the bad means are, in some sense, more
      \emph{psychological} than \emph{technical}.  one usually tend to feel
      ``neglected'' if someone changes someone without announcement, or
      adds a ``better'' version of the file, and this is aggravated by the
      tendency \emph{not to read} the Readme's etc. (``I undertand what I
      want, but what the heck has \texttt{XXX} checked in here, well, I
      don't care/I don't understand, mine was good, too, I go ahead'').
    \end{itemize}
  \item \textbf{Structure of the repos:} Also here, it starts getting out
    of hand. we see signs of bad structure, for instance \emph{names} as
    directory identifiers. It is good to have clean areas of
    responsabilities, but I tend to think it as not a good sign of
    structuring to see it reflected by name in the ``code''.\footnote{At
      the server side, it might be tolerable, but the ``user version''
      should not have to deal with packages called ``Alice'' and ``Bob''.}
    It could again be some polite way of avoiding communication and current
    conflicts (``Let me do my stuff in ``Alice'' and so I leave you in
    peace in your directory in ``Bob'', I don't even look). If it works
    like this, it's ideal, but probably, our situation is not as clean.
    
    Related to the above problems that we see with some of the means of
    communication, we must insist of \textbf{keeping the repos clean!} 
    By this I mean
    \begin{itemize}
    \item no \textbf{garbage}/no \textbf{duplicates}/no \textbf{stale
        code}: it must for the people affected, be clear at (almost) each
      point, what is there, what is the current version that is relevant
      etc.  Of course, it's impossible (and we should not even try) that
      everyone knows everything what's there all the time.  But: if
      something affects others, those concerned must be ``warned'' in
      advance before something is added/changed etc. This warning need not
      be an email or long thread, it can mean \emph{sitting together}
    \item clear separation of own source and third-party stuff (if we have
      some such).
    \end{itemize}
\end{itemize}






\end{document}

%%%%%%%%%%%%%%%%%%%%%%%%%%%%%%%%%%%%%%%%%%%%%%%%%%%%%%%%%%%%%
%% $Id: handout3.tex 23 2004-11-11 07:22:29Z ms $ 
%%%%%%%%%%%%%%%%%%%%%%%%%%%%%%%%%%%%%%%%%%%%%%%%%%%%%%%%%%%%%

%%% Local Variables: 
%%% mode: latex
%%% TeX-master: t
%%% End: 

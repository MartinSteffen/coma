

\newif\ifweb\webfalse

\documentclass[11pt,handout]{handout}

\dozent{Marcel Kyas, Gunnar Schaefer Martin Steffen}
\coursename{P-I-T-M: Coma}
\semester{WS 2004/05}
\ausgabetermin{16.~November 2004 (prerelease 15.11)}
\handouttitle{F}
\nummer{5}
\thema{Group architecture + next steps}





\newcommand{\Coma}{\textsl{Coma}}
\usepackage{hevea}
\usepackage{hyperref}
\usepackage{psfrag}

\usepackage{graphicx}
\graphicspath{{../figures/}} %% cool!


%\uebungfalse
%\handouttrue

\begin{document}

\thispagestyle{empty}





\begin{abstract}
  This handout serve to prepare the decision finding, in particular
  concering the ``group-architecture'' for the rest of the semester. We
  propose a certain group-structure as base for discussion, and give
  arguments and motivations about that structure. Consider the structure
  and think whether it makes sense from your perspective. If not, give
  arguments/alternatives.
\end{abstract}

\section*{Where are we?}



What's needed now is to keep on rolling (or for some to get rolling
\ldots). The \emph{first phase} is \emph{finished} which gave some
clarification about the following points

\begin{itemize}
\item how many people still on board (21?)
\item what tools we choose.
\item some more insight on the specification, in particular the data model.
\end{itemize}

As a further comment on what happened so far is, that the spec-groups had
the tougher task as compared to the tools groups.


\subsection*{Status (Monday evening)}




\begin{description}
\item[org:] Snert is up and running (apache, tomcat, buzilla,
  email-exploder, svn, passwords are handed out); some proposal for the
  next steps is worked out
\item[tools:] we have statements about tools etc from the tools groups (in
  written form only by Alexander Derenbach). So far (Monday evening), the
  tools groups have not yet tested their proposals on snert; they also
  don't have logins.
\item[spec:] we have one (preliminary) specs by today.
\item[test:] we probably got one proposal from the test-groups
\end{description}




\subsection*{Evaluation}

As a \emph{critical review} I'd say:


\begin{itemize}
\item The \emph{specifications} are helpful and necessary (as far as we
  have seen so far). The fact that we have \emph{two} is helpful insofar as
  we could see ideas and get inspiration and discussion from both. As we
  have seen, the work that pushed the semester ahead so far was the spec
  work, so having only one spec group would have left more people ``idle''.
  Furthermore, it forced the developers to \emph{think} and \emph{discuss}
  the tool.
\item Concerning the tools-specification, the success and the usefulness
  (in the context of the project) is less transparent. This has various
  reasons
  \begin{itemize}
  \item the task was perhaps more fuzzy
  \item it's difficult to \emph{see} the success. If the thinking about the
    platform s allowed us to avoid a definite wrong choice, even then the
    success is not visible (because we never know what would have been).
  \item Another important goal in this phase was, that the people proposing
    the tools \emph{are} or \emph{become} experts in
    using/installing/maintaining the tools. From the organizer's
    perspective, we don't know to which extent this has happened. Also this
    depends on how good the ``deliverable'' of ``Getting started with Tool
    xxx'' etc will be.
  \end{itemize}
  On the other hand: it's easy to say: ``why haven't we decided this at the
  beginning'' because in hindsight each decision could have taken
  immediatly (if one had known in advance what it was\ldots). For the
  future: this phase should be faster, nonetheless.
\end{itemize}




\section*{Group architecture}


To keep rolling it is important to get something productive (and useful) to
do for everyone for the weeks to come (in particular already the next
week). Things to be \emph{decided} now is the group architecture. In
Figure~\ref{fig:groups}, we make some proposal. The key points are

\begin{itemize}
\item 3 local modules, which work in isolation i.e., in parallel.  Each
  group consists of \emph{6} persons. Considering the general preferences,
  perhaps 2 ``PHP'' groups and one ``Java'' group will be formed.
\item 3 global modules, which are relevant for all particpants. Those are
  \begin{itemize}
  \item testing: \emph{3 persons}
  \item data base: 
  \item requirements/specification
  \end{itemize}
\end{itemize}


Local means that one needs the code of one of the local groups for the full
tool. For instance, the configuration indicated in Figure~\ref{fig:groups}
by the dotted blob consists of the local implementation of \emph{group 1}
together with the 3 global (or shared) modules.

\medskip



Before we mention concrete tasks within this proposal in a bit more detail,
we first clarify the (partially conflicting) design goals of our proposal.
\begin{description}
\item[programming in the (not too) many:] We are, according to our
  statistics, currently \emph{21 people}. All programming in the same group
  will lead to chaos, in particular it will conflict with the goal that
  (more or less) all participants will contribute in the next phase.
  Furthermore, the task at hand is too small for task, so in order to make
  this work and avoid too many complicated interfaces, there would be a few
  ``real groups'' and the rest would do the doc, the manual, etc., which
  would be boring.
  
  We propose therefore \emph{3 parallel, local groups.} They are small
  enough that internal communication is still possible and that one can
  split the task in interesting enough subparts without creating subtasks
  just because we need to have a task for everyone.  Furthermore, the
  parallel work might inspire competition
\item[preferences] We have seen, that the course is split into ``PHP vs.\ 
  Java'' fractions. The split allows that (more or less) everyone gets what
  he likes.
\item[one spec.] We propose, as the course is intended as a \emph{common}
  programming project, that we do not split it cleanly into completely
  independent working groups. Three things should remain common and shared,
  which are
  \begin{itemize}
  \item testing
  \item requirements/docu, and related to that, 
  \item the concrete data representation in SQL
  \end{itemize}
\item[constant progress] Having finished (more or less) the work on
  spec/tools by \emph{today,} it is not adivsable to make further
  specification week(s) concerning the data-model, because this 
\end{description}





{
\psfrag{G1}{group 1}
\psfrag{G2}{group 2}
\psfrag{G3}{group 3}
\psfrag{test}{test}
\psfrag{spec}{spec./requirements}
\psfrag{DB}{DB/data model}

\begin{figure}[htbp]
  \centering
  \includegraphics[clip=true,height=8.0cm]{group-structure}  
  \caption{Groups}
  \label{fig:groups}
\end{figure}



The goals are conflicting, and the architecture is a compromise between
goals mentioned above. A critical point are (as always) the shared modules,
and in our case the data base. The reason is that everything else depends
on that.\footnote{Note that the picture is not intended to mean, that all
  tools \emph{run} concurrently on the same instance of the data base.}
Since a common tool seems a worthwile and interesting goal, we want to
stick to that even if it will cost time and effort.\footnote{Some software
  people claim also that interfaces tend to get better by the pure fact
  that there's more than one client \ldots} It's less critical for testing
and the requirements, obviously.  Nevertheless, built on the work of the
spec-group, we need to hammer out a unified data model concretely (down to
SQL-expressions) \emph{soon-ish.}  This should doable quickly, and not much
discussion should be needed there.



\subsubsection*{Structure of each group}

We propose therefore the following structure and responsibilities for each
group. So currently this still some sort of ``special assignment'', since
we are not yet coding the final product.

\begin{description}
\item[DB:]
  \begin{description}
  \item[task] \emph{one person} of group 1 -- 3 does a realization of the
    (agreed/common) data model in SQL merge them into one specific
    data-model.
  \item[deadline:] 21?
  \item[deliverable:] specification + SQL-code
  \item[profile:] preferably an SQL-specialist (or someone who wants to
    learn something new \emph{very fast}\ldots)
  \end{description}
\item[data model:] 
\item[profile] that's us, the authors of the handout, we might need some
  clarification from you spec people, however
  \begin{description}
  \item[task] merge the spec. delivs into one common and approved
    specification
  \item[deadline] 18.11 (that's damn soon \ldots)
  \item[deliverable:] some textual format (or graphical, let's see.) It
    should be precise enough that the SQL-person can do their work.
  \end{description}
\item[architecture:] the rest of the persons of groups 1 --3 
  \begin{description}
    \item[task:] depending on your choice of tools: work out a
      modularization of the tools into submodules
    \item[deliverable:] presentation + written description
    \item[deadline:] 23.11
  \end{description}
\item[test:] So far, haven't found a task for those. Lucky again \ldots
\end{description}


\end{document}

%%%%%%%%%%%%%%%%%%%%%%%%%%%%%%%%%%%%%%%%%%%%%%%%%%%%%%%%%%%%%
%% $Id: handout3.tex 23 2004-11-11 07:22:29Z ms $ 
%%%%%%%%%%%%%%%%%%%%%%%%%%%%%%%%%%%%%%%%%%%%%%%%%%%%%%%%%%%%%

%%% Local Variables: 
%%% mode: latex
%%% TeX-master: t
%%% End: 



\newif\ifweb\webfalse

\documentclass[11pt,handout]{handout}

\dozent{Marcel Kyas, Gunnar Schaefer Martin Steffen}
\coursename{P-I-T-M: Coma}
\semester{WS 2004/05}
\ausgabetermin{16.~November 2004}
\handouttitle{not needed, only for freeform}
\nummer{5}
\thema{Group architecture + next steps}


\newcommand{\Coma}{\textsl{Coma}}
\usepackage{hevea}
\usepackage{hyperref}
\usepackage{psfrag}

\usepackage{graphicx}
\graphicspath{{../figures/}} %% cool!


%\uebungfalse
%\handouttrue

\begin{document}

\thispagestyle{empty}





\begin{abstract}
  This handout serve to prepare the decision finding, in particular
  concerning the ``group-architecture'' (and partially also the software
  architecture) for the rest of the semester. We propose a certain
  \emph{group structure} as base for discussion, and give arguments and
  motivations about that structure.  Consider the structure and think
  whether it makes sense from your perspective. If not, give
  arguments/alternatives.
\end{abstract}

\section*{Where are we?}



What's needed now is to keep on rolling (or for some to get rolling
\ldots). The \emph{first phase} is \emph{finished} which gave some
clarification about the following points

\begin{itemize}
\item how many people still on board (21?)
\item what tools we choose.
\item some more insight on the specification, in particular the data model.
\end{itemize}

As a further comment on what happened so far is, that the spec-groups had
the tougher task as compared to the tools groups (or took the initiative to
make it so)


\subsection*{Status (Monday evening)}




\begin{description}
\item[org:] Snert is up and running (apache, tomcat, buzilla,
  email-exploder, svn, passwords are handed out); some proposal for the
  next steps is worked out
\item[tools:] we have statements about tools etc from the tools groups (in
  written form only by Alexander Derenbach). So far (Monday evening), the
  tools groups have not yet tested their proposals on snert; they also
  don't have logins.
\item[spec:] we have one (preliminary) specs by today.
\item[test:] we probably got one proposal from the test-groups
\end{description}




\subsection*{Evaluation}

As a \emph{critical review} I'd say:


\begin{itemize}
\item The \emph{specifications} are helpful and necessary (as far as we
  have seen so far). The fact that we have \emph{two} (or three) is helpful
  insofar as we could see ideas and get inspiration and discussion from
  more than one. 
  
  As we seen it, the work that pushed the semester ahead so far was the
  spec work, and so having only one spec group would have left more people
  ``idle''.  Furthermore, it forced the developers to \emph{think} and
  \emph{discuss} the tool.
\item Concerning the tools-specification, the success and the usefulness
  (in the context of the project) is less transparent. This has various
  reasons
  \begin{itemize}
  \item the task was perhaps more fuzzy
  \item it's difficult to \emph{see} the success. If the thinking about the
    platform s allowed us to avoid a definite wrong choice, even then the
    success is not visible (because we never know what would have been).
  \item Another important goal in this phase was, that the people proposing
    the tools \emph{are} or \emph{become} experts in
    using/installing/maintaining the tools. From the organizer's
    perspective, we don't know to which extent this has happened. Also this
    depends on how good the ``deliverable'' of ``Getting started with Tool
    xxx'' etc will be, or how supportive the expert will be if the set-up
    phase runs into troubles.
  \end{itemize}
  On the other hand: it's easy to say: \emph{``why haven't we decided this
    at the beginning?''} because in hindsight each decision could have
  taken immediately (if one had known in advance what it was\ldots). For the
  future: this phase should be faster, nonetheless.
\end{itemize}




\section*{Group architecture}


To keep rolling it is important to get something productive (and useful) to
do for everyone for the weeks to come (in particular already the next
week). Things to be \emph{decided} now is the group architecture. In
Figure~\ref{fig:groups}, we make some proposal. The key points are

\begin{itemize}
\item 3 local modules, which work \emph{in parallel.}  Each group consists
  of \emph{6} persons. Considering the general preferences, perhaps 2
  ``PHP'' groups and one ``Java'' group will be formed.
\item 3 global or shared modules, which are relevant for all participants.
  Those are
  \begin{itemize}
  \item testing: \emph{3 persons}
  \item requirements/specification (the organizers, with help from
    delegates of the spec groups, if needed.)
  \item data base: by special assignment
  \end{itemize}
\end{itemize}


Local means that one needs the code of one of the local groups for the full
tool. For instance, the configuration indicated in Figure~\ref{fig:groups}
by the dotted blob consists of the local implementation of \emph{group 1}
together with the 3 global (or shared) modules.

\medskip



Before we mention concrete tasks within this proposal in a bit more detail,
we first clarify the (partially conflicting) design goals of our proposal.
\begin{description}
\item[programming in the (not too) many:] We are, according to our
  statistics, currently \emph{21 people}. All programming in the same group
  will lead to chaos, in particular it will conflict with the goal that
  (more or less) all participants will contribute in the next phase and
  also throughout the semester. Furthermore, the task at hand is too small
  for 21 people, so in order to make this approximately work and avoid too
  many complicated interfaces, there would be a few ``real groups'' and the
  rest would do the documentation, the manual, a further statistic module
  etc., which could be boring.
  
  We propose therefore \emph{3 parallel, local groups.} They are small
  enough that internal communication is still possible and that one can
  internally split the task in interesting enough subparts without
  introducing subtasks just because we need to have a task for everyone.
  Furthermore, the parallel work might inspire competition.
\item[preferences] We have seen that the course is split into ``PHP vs.\ 
  Java'' fractions. The proposed structure allows that (more or less)
  everyone gets what he likes.
\item[one spec.] We propose, as the course is intended as a \emph{common}
  programming project, that we do not split it cleanly into completely
  independent working groups. Three things should remain common and shared,
  which are
  \begin{itemize}
  \item testing
  \item requirements/docu, and related to that, 
  \item the concrete data representation in SQL
  \end{itemize}
\item[constant progress] Having finished (more or less) the work on
  spec/tools by \emph{today,} it is not advisable to schedule further
  specification week(s) concerning the data-model, because this will just
  produce delay. We must enter a new phase now, with the spec. at hand.
\end{description}



{
\psfrag{G1}{group 1}
\psfrag{G2}{group 2}
\psfrag{G3}{group 3}
\psfrag{test}{test}
\psfrag{spec}{spec./requirements}
\psfrag{DB}{DB/data model}

\begin{figure}[htbp]
  \centering
  \includegraphics[clip=true,height=8.0cm]{group-structure}  
  \caption{Groups}
  \label{fig:groups}
\end{figure}


The goals are conflicting, and the architecture is a compromise between
goals mentioned above. A critical point are (as always) the shared modules,
and in our case the data base. The reason is that everything else depends
on that.\footnote{Note that the picture is not intended to mean, that all
  tools \emph{run} concurrently on the same instance of the data base.}
Since a common tool seems a worthwhile and interesting goal, we want to
stick to that even if it will cost time and effort.\footnote{Some software
  people claim also that interfaces tend to get better by the pure fact
  that there's more than one client so the architecture may lead to better
  end result(s) \ldots} It's less critical for testing and the
requirements, obviously.  Nevertheless, built on the work of the
spec-group, we need to hammer out a unified data model concretely (down to
SQL-expressions) \emph{soon-ish.}  This should doable quickly, and not much
discussion should be needed there. We, the organizers, will take the data
part next and decide what to take from it and what to leave out. 



\subsubsection*{Structure of each group}

We propose therefore the following structure and responsibilities for each
group. So currently this still some sort of ``special assignment'', an also
we are not yet coding the final product.



\begin{tabular}{ll}
  \\\hline
  \textbf{DB} & data base/sql-representation
  \\\hline
  &
  \begin{tabular}[t]{lp{10cm}}
    \textbf{task} &  \emph{one person} of group 1 -- 3 does a realization of the
    (agreed/common) data model in SQL merge them into one specific
    data-model.
    \\
    \textbf{deadline:} & 21.11
    \\
    \textbf{dependencies:} & the data model task
    \\
    \textbf{deliverable:} & spec + SQL-code
    \\
    \textbf{profile:} & 1 person from group 1 -- 3,  preferably an SQL-specialist (or someone who wants to
    learn something new \emph{very fast}\ldots)
  \end{tabular}
  \\\hline\hline
  \textbf{M} & data model
  \\\hline
  &
  \begin{tabular}[t]{lp{10cm}}
    \textbf{task:} & wrap up/unify the spec as far as the data is
    concerned. I.e., merge the spec. delivs into one common and approved
    specification
    \\
    \textbf{profile:} &  that's us, the authors of the handout, we might need some
    clarification from you spec people, however
    \\
    \textbf{deadline:} & 18.11 (that's damn soon \ldots)
    \\
    \textbf{deliverable:} & some textual format (or graphical, let's see.) 
    It should be precise enough that the SQL-person can do their work.
    \\
    \textbf{dependencies:} & in the Spec task of the first phase
    \\
  \end{tabular}
\end{tabular}


  \begin{tabular}{ll}
  \textbf{A} & architecture 
  \\\hline
  &
  \begin{tabular}[t]{lp{10cm}}
    \textbf{task:} & depending on your choice of tools: work out a
      modularization of the tools into submodules and subtasks, make a plan how they
      reside  in the  repository.  Furthermore make a proposal of \emph{assignment}
      of the workload internally in the group, i.e., assign
      responsibilities internally. 
      \\
      \textbf{profile:} &  the rest of the people of the group.
      \\
      \textbf{deadline:} & 23.11 
      \\
      \textbf{deliverable:} & written document addressing the questions above.
      \\
      \textbf{dependencies:} & no real ones, depends on the chosen set of
      tools, however.
      \\
    \end{tabular}
    \\\hline\hline
    \textbf{T} & tools
    \\\hline
    &
    \begin{tabular}[t]{lp{10cm}}
      \textbf{task:} & depending on your choice of tools: get yourself
      accustomed to \texttt{snert.} Get yourself enough privileges there
      (whatever is needed), make it so that all the tools are (really)
      available (and not just ``in principle it's there''). Make sure that
      everyone of your group has the required software also at his
      workplace (whatever it is).
      \\
      \textbf{profile:} &  1 dedicated person in each group. Preferably one
      from the tools task force of the previous phase.
      \\
      \textbf{deadline:} & 23.11 
      \\
      \textbf{deliverable:} & written document, preferably short, that
      enables the co-workers of your group + the organizers to use the
      tools. Explicit statement about the versions of the tools, platforms
      etc, that your software relies on.
      \\
      \textbf{dependencies:} & no real ``vertical'' one except the tools
      phase. Some horizontal coordination, however, is required. If two
      groups want to use PHP, both should agree on \emph{the same}  version
      as default etc.
      \\
    \end{tabular}
    \\\hline\hline
    \textbf{Q} & testing/quality assurance
    \\\hline
    &
    \begin{tabular}[t]{lp{10cm}}
      \textbf{task:} & extend your test-concept to the proposed group
      structure (concept + tool-wise)
      \\
      \textbf{profile:} &  3 persons, preferably the test-guys from the first
      phase. 
      \\
      \textbf{deadline:} & 23.11 
      \\
      \textbf{deliverable:} & written document answering the questions above, stating also some policy of
      testing, rule etc. Install 
      \\
      \textbf{dependencies:} & no real, depending however on the work of the
      test group in the first phase.
      \\
    \end{tabular}
  \end{tabular}

\end{document}

%%%%%%%%%%%%%%%%%%%%%%%%%%%%%%%%%%%%%%%%%%%%%%%%%%%%%%%%%%%%%
%% $Id: handout3.tex 23 2004-11-11 07:22:29Z ms $ 
%%%%%%%%%%%%%%%%%%%%%%%%%%%%%%%%%%%%%%%%%%%%%%%%%%%%%%%%%%%%%

%%% Local Variables: 
%%% mode: latex
%%% TeX-master: t
%%% End: 

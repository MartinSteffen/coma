\newif\ifweb\webfalse

\documentclass[11pt,nologo,handout]{handout}

\dozent{Marcel Kyas, Gunnar Schaefer Martin Steffen}
\coursename{P-I-T-M}
\semester{WS 2004/05}
\ausgabetermin{9.~November 2005}
\handouttitle{F}
\nummer{3}
\thema{Bugzilla}

\newcommand{\Coma}{\textsl{Coma}}
\usepackage{hevea}
\usepackage{hyperref}

%\uebungfalse
%\handouttrue

\begin{document}

\thispagestyle{empty}

The purpose of this handout is to explain the purpose and how to use
bugzilla.  Bugzilla is a web-based bug-tracking system.  In the following,
we first explain the purpose of bug-tracking, the anatomy and the
life-cycle of a bug in bugzilla, how bugzilla is used, and finally the
policies of using bugzilla in this project.


\section{Bugtracking}

The main functionality of bugzilla is maintaining a data base of
bug reports and track the state of bugs.  The main purpose of
bugzilla is to help in \emph{collaborating} in correcting bugs.
It helps to keep track of the status of an error, for instance:
\begin{itemize}
\item who reported the error?
\item what kind of error, short description?
\item who is (probably) responsible?
\item was it acknowledged?
\item has it been repaired?
\end{itemize}

Bugzilla can generate statistics and reports on open bugs, who is
responsible for a bug, and in what state it is.  It can tell you
how many bugs and which ones have to be corrected to reach a mile
stone.  But similar to {\Coma} it does not take decisions and it does
not correct bugs.

Bugtracking is a necessity in software development.  It helps in
classifying what has to be done when.  It keeps people informed on
what needs to be done.  It simplifies the process of figuring out
who is responsible for correcting a bug.

Bugzilla is a tool to make life in software development simpler.





\section{Anatomy of a bug}

Bugs are modeled by the following fields.
\begin{description}
\item[Product and Component]  Bugs are divided up by Product and Component,
  with a Product comprising many Components.  In our case the product is
  ``{\Coma} conference manager''.  Currently, the components of coma are
  \begin{description}
  \item[Data Model]  This component represents the \emph{data model} of
    {\Coma}.  The data model describes which data is represented within
    the tool and how it is related to each other.
  \item[Specification]  This component represents the \emph{specification}
    document.  Bugs filed against this component may include enhancements
    to the functionality, inconsistencies in the document, or suggestions
    on how to change the specification to meet the overall goal within the
    tight time constraints.
  \end{description}
  Further components will be added once the architecture of the tool has
  been settled.  Each component has an initial owner and an
  initial quality assurance (QA) contact.  The \emph{initial owner}
  is the person who is automatically responsible for analysing and
  correcting this bug.  The \emph{initial QA contact} is responsible
  for verifying that the resolution of a bug is indeed a correct once.
  More details later.
\item[Status and Resolution]  These define exactly what state the bug is in
  --- from not even being confirmed as a bug, through being corrected and
  the correction being confirmed by QA.
\item[Assigned To]  The person responsible for fixing the bug.
\item[URL] A URL associated with the bug, if any.  This can be used to
  associate external documents to a bug without attaching it to them.
  It can, e.g., point to a section of the requirements or specification
  document which explains why this is a bug.
\item[Summary] A one-sentence summary of the problem.  This summary should
  be meaningful.  ``It does not work'' is not meaningful.  ``Uploading an
  article does not send a confirmation'' is.
\item[Status Whiteboard]  A free-form text area for adding short notes and
  tags to a bug.
\item[Keywords]  This is currently not used.
\item[Platform and OS]  These indicate the computing environment where the
  bug was found.
\item[Version]  This field usually contains the particular version of the
  product in which the bug was found.  In our case it is the same as
  milestones.
\item[Priority]  The bug assignee uses this field to prioritise his or her
  bugs.  It's a good idea not to change this on other people's bugs.  In
  {\Coma} priorities range from P1 (the highest value) to P5 (the least
  value).  Bugzilla defaults to P2 for each new bug.
\item[Severity]  This indicates how severe the problem is.  Values are
  \begin{description}
  \item[blocker]  This error makes the application unusable.
  \item[critical]  The error is a security error or an error which does not
    yet make the application unusable but it misbehaves in a way which
    violates the specification.  E.g., if an author can see the discussion
    on his paper, this would be a critical bug.
  \item[major]  The error does not pose any danger to the integrity to the
    application, but it seriously impedes usability.  If uploading a paper
    silently fails, this would be a major bug.
  \item[normal]  It is an error that needs to be corrected, but there is
    a work-around to achieve this functionality.
  \item[minor]  It is an error which does not affect the main functionality
    of the application.  Most errors in error-paths (exceptions handled in
    an unfortunate way) are of this kind.
  \item[trivial]  These are cosmetic issues.  Spelling errors or errors in
    comments are trivial.
  \item[enhancement]  These are not bugs but enhancement requests.  A valid
    enhancement request would be to ask another group to export (define an
    interface) some functionality, which otherwise has to be implemented on
    your own.  Also the definition of new views can be an enhancement.
  \item[tracking]  These are not bugs, but meta-bugs.  The idea of a
    tracking bug is that it collects all bugs which need to be corrected
    for a certain feature, or even tracks the discussion on the
    implementation of a certain feature.  The use of this severity is
    strictly optional.
  \end{description}
\item[Target]  A future version by which the bug is to be corrected.
  Values are:
  \begin{description}
  \item[M0]
  \item[M1]
  \item[M2]
  \item[M3]
  \item[Presentation]  The final day of the lab course where the product
    is to be presented.
  \end{description}
\item[Reporter]  The person who filed the bug.
\item[CC list]  A list of people who get mail when the bug changes.
\item[Attachments]  You can attach files (e.g. test cases or patches) to
  bugs.  If there are any attachments, they are listed in this section.
\item[Dependencies]  If this bug cannot be corrected until other bugs are
  corrected (depends on), or this bug stops other bugs being fixed (blocks),
  their numbers are recorded here.
\item[Votes]  Whether this bug has any votes.  This is not used in our
  lab course.
\item[Additional Comments]  If you have something worthwhile to say, you
  can add it to this section.  In this lab course, each change of a bug
  requires the changer to add a \emph{meaningful} comment on why he
  changed the bug.
\end{description}

The most important part on understanding a bug is its state and how a bug
changes its state.  It is graphically depicted in
Figure~\ref{fig:lifecycle}.  A state may have a \emph{resolution} as its
substate.
\begin{description}
\item[Open]  Every new bug starts its life in the open state.
\item[Assigned]  If a bug has been analyzed and found to be a valid bug
  report, then the state is changed to assigned.  This step entails
  confirming that the bug is filed against the right component, that it
  is reproducable, and naming the person responsible for correcting the
  bug.
\item[Resolved]  Once a bug is corrected or the report is inappropriate,
  the state is changed to resolved.  This state also allows one to set
  the resolution.  This is a value from:
  \begin{description}
  \item[Fixed]  The error has been corrected.
  \item[Invalid]  The bug report is considered to be invalid.  This
    is used if the reporter uses a function of another component which
    is not exported in the interface in a way which is wrong, expects
    functionality of an exported function that is not specified or
    which contradicts the specification, or if the report is not meaningful
    at all.  A bug with the summary ``The tool is crashing when it is
    raining'' is definitely invalid, as is ``x*x does not return negative
    values''.
  \item[Wontfix]  The responsible person will not correct the bug.  You
    can use this resolution, if you think that the behavior described in
    the report is indeed a bug, but there is either a workaround, or the
    reporter used a functionality in an unintended way.  Another
    appropriate situation is, when a client calls a function of your code
    that is not part of the exported interface of your component.
  \item[Remind]  This resolution can be used for bugs with a low severity
    or a low priority.  It is used for errors which can be shipped in a
    final product and essentially means: ``I'll correct it at the end of
    the project, if I have time''.
  \item[Worksforme]  This resolution is used for bug reports which cannot
    be reproduced.
  \end{description}
\item[Verified]  Once an error is resolved, the QA responsible has to check
  whether the resolution is correct.  If so, he changes the state of the
  bug to verified.  Note that the resoltion set in resolved also is part
  of the verified state.
\item[Closed]  If everybody is happy with the resolution, the state of a
  bug is changed to closed.  This means that the bug is not present in the
  software anymore.
\item[Reopened]  If a resolution is found to be invalid or a bug which has
  been resolved appears again, a bug can be reopened.  For all practical
  purposes this state is the same as open.
\end{description}

\begin{figure}
  \centering
  %% Put figure here.
  \caption{A bug's life cycle}
  \label{fig:lifecycle}
\end{figure}






\section{Bugzilla}

\href{http://www.bugzilla.org}{Bugzilla} is some server which helps to
maintain a common database of bugs during the software devolopment.  The
relevant information on using bugzilla is explained
in~\href{http://www.bugzilla.org/docs/2.18/html/using-intro.html}{Chapter 5}
of~\href{http://www.bugzilla.org/docs/2.18/html/}{bugzilla's user manual}.

In order to use bugzilla, it is necessary to create a user account
\emph{with a working email address}.





\section{Policies}

Bugzilla can be a helpful tool when used with care, but obviously it does
not guarantee solid development.  It is necessary to understand that
bugzilla is here to help developing and not replace developing software.
Everybody has to understand that the purpose of bugzilla is not to
report bugs and track bugs,\footnote{Well, this actually is, what it does}
but to \emph{arrive with a working product}.  To achieve this end, we
propose the following policies.

If you want to report a bug, first think about whether it is really a
bug and in which components.  Not all bugs are in the code.  Some may
be in the documentation or even the specification.  The first step when
writing a bug report is to decide what the nature of the bug is and in
which component you suspect the error.

Then think about the text of the report.  Give a description of the bug
which allows anybody else to reproduce the bug.  Write down what you did,
what you expected as a response, and what you actually got.  If such a
description is missing or erroneous, the assignee is free to resolve
the bug as invalid or worksforme.  Bug reports with patches are even
better.

Remember, writing a good bug report takes time.  Writing reports with
patches is even better.  Writing bad ones will just undermine the
purpose of bugzilla and propably kill the project.

Keep the bug data base up to date.  More importantly, \emph{collaborate}
in keeping the bug data base up to date.  If you see a bug report which
appears to be solved, ask whether it has been resolved or resolve it
yourself with worksforme and a comment that, when you tested the code,
it worked.  If you see duplicate reports, feel free to mark them as
duplicates.  If you know something helpful, comment on the bug.

Whenever you change the state of the bug, write a meaningful comment
on why you changed it.  It needn't be a long comment.

Nobody should change the priority or severity of a bug without very
good reasons!

If a bug is considered trivial or minor, delay its resolution.  To do
this, set its state to resolved with resolution remind.  This allows
us to see that you do not feel that it is necessary to correct the bug
right now.

Use bugzilla only for discussing details of bugs.  Do not put everything
you do into bugzilla.  We have introduced tracking bugs into the
data base, but this is only useful for certain dangerous changes or very
complex tasks which need much communication.

Do not push for a resolution.  Often it is useful to delay the resolution
in order to have something useful for a milestone.  Often, the right
solution is to reassign a bug.  But we should agree on these steps.

And finally: If you like to go bug hunting in other peoples code instead
of working on your own, we assume that you have too much time and expect
you to help that group correct these bugs (just kidding).  On the other
hand, a bug report may be assigned to the reporter in certain
circumstances.

\end{document}

%%%%%%%%%%%%%%%%%%%%%%%%%%%%%%%%%%%%%%%%%%%%%%%%%%%%%%%%%%%%%
%% $Id$ 
%%%%%%%%%%%%%%%%%%%%%%%%%%%%%%%%%%%%%%%%%%%%%%%%%%%%%%%%%%%%%

%%% Local Variables: 
%%% mode: latex
%%% TeX-master: t
%%% End: 

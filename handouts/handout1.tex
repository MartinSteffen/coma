%\newif\ifsolution\solutionfalse
%\newif\ifworking\workingfalse

\newif\ifweb\webfalse

\documentclass[11pt,nologo,handout]{handout}


\dozent{Marcel Kyas, Gunnar Schaefer Martin Steffen}
\coursename{P-I-T-M}
\semester{WS 2004/05}
\ausgabetermin{9.~November 2005}
\handouttitle{F}
\nummer{2.(1)}
\thema{Platform and software}

\newcommand{\Coma}{\textsl{Coma}}


%\usepackage[english, german]{babel}

\usepackage{hevea}
\usepackage[latin1]{inputenc}
\usepackage{hyperref}

%\uebungfalse
%\handouttrue

%%% common macro files for the projects


\newcommand{\NONAME}{Somename}


\newcommand{\Coma}{\textsl{Coma}}

\newcommand{\cvs}{cvs}
\newcommand{\Java}{\textsc{Java}}
\newcommand{\Cplusplus}{C$^{++}$}
\newcommand{\javadoc}{\textsc{javadoc}}



\newcommand{\team}[1]{\textbf{Responsible:} #1\bigskip{}}


\newcommand{\bnfdef}               {::=}
\newcommand{\bnfbar}               {\mathrel{|}}

\newcommand{\suchthat}{\mathrel{\mid}}
\newcommand{\union}  {\cup}
\newcommand{\intersect}  {\cap}
\newcommand{\sizeof}[1]  {{\mid}{#1}{\mid}}
\newcommand{\sem}    [2]        {[\![#2]\!]_{#1}}      %semantics

\newenvironment{diagram}{\begin{displaymath}}{\end{displaymath}}


%%% Local Variables: 
%%% mode: latex
%%% TeX-master: t
%%% End: 

\begin{document}

%\ausgabe{19.~Oktober 2004}
%\maketitle{x}{Koordination mit CVS}


\thispagestyle{empty}


The purpose of this handout is to give infomration about the available
platforms, access methods, the currently available software.


\section{Hardware + access}

We have one (not-too fast, not-too-big, not-too-new \ldots) PC hooked up to
the net, over which we have control. Even it the machine is not a
high-performance server,  the tool should run

Here is some data descibing the machine.

\begin{table}[htbp]
  \centering
  \begin{tabular}{ll}
    name & \texttt{snert.informatik.uni-kiel.de}
    \\
    IP &  134.245.253.11
    \\
    disc & ca. 3G available (/home)
    \\
    Mhz & 800Mhz
    \\
    main memory & 128M
    \\
    swap & ca 256M
    \\
    OS & Linux 2.6.8 / Fedora core 2 (+ updates)
    \\
  \end{tabular}
  \label{tab:platform}
\end{table}

The purpose of the machine is 3fold\footnote{One particular purpose for
  which we cannot use the machine directly is for \emph{backup,} since it
  is not part of the network's backup system. As solution, we are running a
  cronjob which in regular intervals copies relevant data to the sun-pool.}
\begin{enumerate}
\item subversion server
\item test platform
\item bugzilla server
\end{enumerate}

More details concerning in particular subversion that below. 

%But in order to access the machine, we
%need from everybody a separate \emph{password} for subversion access (via
%apache) i.e.  everyone must physically appear at the machine and pick a
%password.


\section{Subversion}
\label{sec:subversion}

From the discussion, we learnt that we will use \emph{subversion}, which is
mandatory for all participants.  Currently, we have on the solaris pool the
\emph{server side} of subversion running.\footnote{The reason why we can't
  us a solaris machine also for the server side, which would simplify
  matter a bit is technical: currently subversion cannot deal with NFS'ed
  file systems very well, and the sys-admin are currently not (yet) setting
  up a dedicated svn-server machine.}

The command is
\begin{quotation}
  \texttt{/opt/local/bin/snv}
\end{quotation}
which (probably) means, it's in your execution path. Later we will discuss
further rules of the game concerning access to the server, but here are a
few easy commands to start with. For deeper knowledge, refer to the
subversion manuals. Their are linked to our course pages.



\subsection{Technical prerequisites}

To access the repos, you need the client-side part of \emph{subversion}.
For the solaris pool this is installed (see below). Subversion is also
shipped in modern Linux distributions. As far as installation on Windows or
Mac platforms is concerned, there's no experience from our side, but it
seems that some of the participants have expertise there. If proplems
should arise there, please share the knowledge.


Furthermore, for write access to the repos, you need some \emph{password.}
We will send more instructions about this later.






\subsection{First access and survival guide}


Luckily, many commands look very similar to the ones you might be used from
cvs. The most important ones are probably:
\begin{description}
\item[checking out], i.e., getting the first copy of the repository into
  ones own workspace
\item[update] getting one's own copy into sync with the general development
\item[commit] writing back own changes
\end{description}



\subsubsection{Checkout}

For a read access, checking out the currently rather empty repository, you
don't yet need a password. Create (or choose)in your workspace some
appropriate directory, say \texttt{WORKDIR}, and then do


\begin{verbatim}
   cd <WORKDIR>
   svn co http://snert.informatik.uni-kiel.de:8080/svn/coma 
\end{verbatim}

As an explanation: \texttt{co} stands for ``checkout'' (also
\texttt{checkout} instead would do), next comes the url of the
\emph{repository} which is accessed via http and the apache server running
on snert. The part \texttt{/svn/coma} gives the \emph{project}
\texttt{coma} on the repository which is specified by
\texttt{/snv/}.\footnote{The \texttt{/svn/} is one full repository as seen
  through apache. At the server side, the actual directory containing the
  repository is kept at a space in the file system accessible by apache.}

%The last argument \texttt{coma}
%give the client side (sub-)directory, in which the project is developed. We
%chose it to have the same name as on the server which is plausible, but not
%mandatory.







\subsection{Testing}

If new to subversion or and version control, it might be help to play
around a bit. In order to do this without fear of destroying real work, we
set up a \emph{second repository} for that purpose. It is completely
separated from the ``real'' one, so it's not just another project on
\texttt{/svn/} but is called \texttt{/snvtest/} instead.

So you might try:

\begin{verbatim}
   svn co http://snert.informatik.uni-kiel.de:8080/svntest/coma comatest
\end{verbatim}




\subsection*{Example}

The example uses \texttt{\$WORKDIR} as name for the \emph{working
  directory} you choose appropriately. In principle you can have as many
working directories at the same time, i.e., more than one checked out copy
of your project. A typical scenario would be to have one at the university
pool and one at the workplace at home. With the number of checked out
working directories, however, the danger of confusion increases; in normal
situation, one woring directory per ``workplace'' suffices.
\begin{description}
\item[check out:] as explained already above, the following gives you
  the first copy of the \Coma-project:
\begin{verbatim}
         mkdir $WORKDIR
         cd $WORKDIR
         svn co http://snert.informatik.uni-kiel.de:8080/svn/coma 
\end{verbatim}
  As said: Instead of \texttt{svn}, you might wish to play around with
  \texttt{svntest} for a start.  The ``variable'' \texttt{\$WORKDIR} is
  here just for illustration, pick yourself an appropriate place.  You can
  also pick only parts of the repos, for instance

\begin{verbatim}
        svn co http://snert.informatik.uni-kiel.de:8080/svn/coma/org
\end{verbatim}
  gives you info about the groups etc.
\item[write back:] Once done with your changes, you can hand back the whole 
  work tree, doing
\begin{verbatim}
              cd $WORKDIR/coma
              svn commit 
\end{verbatim}
  Afterwards you will be asked to give a comment about what changes you did in some editor.
  If you do not have much to say for that revision, you can write back faster by typing
\begin{verbatim}
              cd $WORKDIR/coma
              svn commit -m"Small bug fix"
\end{verbatim}
\item[update:] to get the tree in your workspace in sync again, i.e., to
  obtain possible new changes, is done by:
\begin{verbatim}
               cd $WORKDIR/coma
               svn update
\end{verbatim}
  Typically, this should be done when you start working \ldots
\item[new files and directories] can be added by
\begin{verbatim}
               svn add [filename]
\end{verbatim}
  They can be removed again by
\begin{verbatim}
               svn remove [filename],
\end{verbatim}
  Instead of a filename, also a directory can be added. Furthermore,
  subversion honors regular expression, so it understands things like
  \texttt{svn add *.java}.
\item[moving \& copying:] Here's something new for cvs-users: you can move
  really move directories (and files). The fact, however, that this is
  possible does not mean, that it is useful to keep the project under
  constant re-construction. Cf. the remarks about policy below.
\end{description}

\subsection*{Strategy and policy}

Version control does not guarantee smoothless work. I supports it, however,
if used with with one's brain switched on and with discipline. From out
side, we have not yet had experience with subversion (with CVS however),
therefor more specific do's and don't may follow. The following rules of
thumb might give first guidance.

\begin{itemize}
\item In general: if there is danger that changes affect negatively other
  group's work, clarify this before you check in your changes.
  In particular
  \begin{itemize}
  \item directly changing other persons's work (``I can fix that
    better/faster than those \ldots''): \emph{only} after \emph{careful
      reflection} and communication with the concerned group
  \item No unannounced change in the directory structure. New
    subdirectories in one's own part are ok, though. No fiddling around
    with administrative  subversion properties.
  \item No global \emph{undo} (``quick fix repair'') of other person's
    changes without communication.
%  \item Keine ``\emph{watches}'' setzen (au"ser eventuell auf seinen
%    eigenene Code).
  \end{itemize}
\item Later in the course: check in only versins that \emph{integrate} with
  the rest of the code (for instance, the can compile together). We will
  work out a strategy for that, probably using \emph{make}. After
  intergration interface changes (whether the interfaces are formal like
  Java interfaces or informal such as common agreements) \emph{must be
    announced} and accepted by concerned parties. (for instance during the
  meetings or the email list).
\item \texttt{Makefile}s and \texttt{Readme}s are useful.
% Der Code f"ur
%  jedes Pakets soll mittels \texttt{make all} als erstes ``target'' des
%  Makefiles kompilierbar sein. Im Paket \texttt{absynt} finden sich
%  Beispiele.
%\item Der Code soll sinnvoll mittels \textsl{Javadoc} kommentieren werden.
%  Zumindest der Author/die Autoren sollen dokumentiert sein.  In Paket
%  \texttt{absynt} kann man sich anschauen, wie man auch cvs-logs
%  dokumentieren lassen kann, (z.B.\ indem man die Datei
%  \texttt{templates/classtemplate.txt} verwendet und anpa�t.) Die
%  generierte Dokumentation wird in regelm"a"sigen Abst"anden im Netz
%  bereitgestellt.
\end{itemize}


\section{Further software}

There is a number of other software installed on the machine. we give here
a short overview, a later handout will contain an update, when new stuff
arrived, but also more specific infos concerning the use and the access.



\subsection{Bugzilla}


\href{http://www.bugzilla.org}{Bugzilla} is some server which helps to
maintain a common database of bugs during the software devolopment. In
particular, it helps to keep track of the status of an error, for instance 
\begin{itemize}
\item who reported the error?
\item what kind of error, short description?
\item who is (probably) responsible?
\item was it acknowledged?
\item has it been repaired?
\end{itemize}

As with subversion, it can be be a helpful tool when used with care, but
obviously does not guarantee solid development.  For instance, reporting
bugs and having a nice overviews over the various developments does not
help in the end, if noone feels responsible for fixing them \ldots In
previous courses, we used a ``manual'' kind of bug-tracking system and that
was pretty much the situation. 

With bugzilla such much more powerful and easer to use than some textual
(but formatted) \emph{error-lists}, different dangers might occur, for
instance that we are swamped in error logs and we loose the focus. We might
hunt down one tiny bug after the the other, the statistics looks nice, but
we loose in ther overall goal.

A further danger could be social. People don't like if someone else spot
errors in one's own work (and make them public, as well). A reported error
as such is first of all something useful, but don't take it as
``\emph{sport}'' to notch as many (big, small, miniscule) errors of others
on your knife\ldots. And refrain from error reports as \emph{retaliation!}
(``Now if they find 5 bugs in my part, let's sit down and see what kind of
shitty code they have, I'll give them back 50, that should teach them''.
yes, things happen)


\medskip

Since we currently don't know, how people will adapt to the situation (till
next week, we might hand out further hints concerning rules of engagement)
we will have to monitor whether bugzilla is used to support the aims of the
project.  Reacall: bugzilla is a \emph{tool} not a \emph{goal} of the
course.




\begin{table}[htbp]
  \centering
  \begin{tabular}{ll}
    php & 4.3.8 
    \\
    python & 2.3.3
    \\
    mySQL & 3.23.58
    \\
    apache httpd &
    \\
    ``tomcat'' & partly installed
    \\
    ``java'' &  1.4
  \end{tabular}
  \caption{further software @ snert}
  \label{tab:software}
\end{table}









\nocite{subversion:manual}
%\nocite{cvs:www}
%\nocite{cvs:manual}
%\nocite{fogel:cvs}


\bibliographystyle{alpha}

\addcontentsline{toc}{section}{Bibliography}


{%\small
  \bibliography{string,etc,imperative,oop,types,protocol,functional,logic,crossref}
}

\end{document}



\subsection*{Technische Vorraussetzungen}

Man ben�tigt zwei Dinge, um von Ferne auf das gemeinsame Repositorium
zugreifen zu k�nnen:
\begin{enumerate}
\item \textsl{ssh} (secure shell) f"ur den sicheren Zugriff und
\item \textsl{cvs} + \textsl{rcs} als Versionskontrolle.
\end{enumerate}


Diese Tools sind auf dem Uni-Netz vorhanden. Wer daheim (oder von einem
sonstigen account) arbeiten will, mu"s daf"ur sorgen, da"s sie installiert
sind. Sie sind kostenlos f"ur Linux/Unix/Windows/MacOS verf"ugbar.



\subsubsection*{Angabe des cvs-Servers}

F�r das korrekte Funktionieren mu� man cvs noch mitteilen, wo sich das
Repositorium befindet. Dies kann man wie folgt erreichen (bash-Syntax):

\begin{verbatim}
    export CVSROOT=swprakt@goofy.informatik.uni-kiel.de:/home/swprakt/cvsroot 
    export CVS_RSH=ssh
    export CVSEDITOR=emacs

    export CLASSPATH=$WORKDIR/coma/src:
\end{verbatim}



wobei \texttt{\$WORKDIR} ein Platzhalter (!) f�r das Verzeichnis ist, in
dem man arbeiten wird. Die ersten drei Zeilen sind zum korrekten Steuern
von cvs gedacht.\footnote{Man kann cvs auch anders steuern, z.B. �ber
  .cvsrc, bei Bedarf bitte selbst im Manual nachschlagen oder nachfragen.}
Anstelle \textsl{emacs} kann man auch \textsl{emacsclient} oder den Editor
seiner Wahl nehmen.  \textsl{emacsclient} sollte man dann nehmen, wenn man
ohnehin emacs verwendet und beim starten des emacs' ein
\texttt{(server-start)} ausgef�hrt wird.  Die letzte Zeile hat nichts mit
ssh/cvs, sondern hat mit dem diesj"ahrigen Java-Projekt selbst zu tun.




\section*{Verteiltes Arbeiten unter CVS}

Zu Koordination der einzelnen Gruppen ist die Verwendung von
``\textit{Concurrent Version Control''} (CVS) vorgeschrieben. Wir werden
eine kurze Einf�hrung in CVS geben und einen \textit{account} als Server
f"ur das \emph{Repositorium} zur Verf�gung stellen. Die wichtigesten Dinge
sind auf diesem Handout zusammengefa�t. \ifweb\else Weitere Informationen
�ber unsere Webseite. \fi






\subsubsection*{ssh-Zugriff}



F�r den ersten Punkt, dem Fernzugriff unter \textsl{ssh}, geht man wie
folgt vor.  Man beschaffe sich einen \textsl{ssh}-Schl�ssel Der Schl�ssel
liegt typischerweise unter \texttt{\home{}/.ssh/identity.pub}.  Diesen
maile man an \texttt{swprakt@informatik.uni-kiel.de}. Wer einen solchen
Schl�ssel nicht hat, mu� ihn sich generieren und zwar mittels
\textsl{ssh-keygen} am besten \texttt{ssh-keygen -tdsa} f�r einen ssh-2
Schl�ssel.



%%%%%%%%%%%%%%%%%%%%%%%%%%%%%%%%%%%%%%%%%%%%%%%%%%%%%%%%%%%%%
%% $Id: handout1.tex,v 1.1 2004/11/08 17:05:55 swprakt Exp $ 
%%%%%%%%%%%%%%%%%%%%%%%%%%%%%%%%%%%%%%%%%%%%%%%%%%%%%%%%%%%%%


%%% Local Variables: 
%%% mode: latex
%%% TeX-master: t
%%% End: 

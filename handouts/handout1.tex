\newif\ifsolution\solutionfalse
\newif\ifworking\workingfalse
\newif\ifweb\webfalse



\documentclass[11pt,nologo,handout]{handout}


\dozent{Marcel Kyas, Martin Steffen, Gunnar Sch\"afer}
\coursename{P-I-T-M}
\semester{WS 2004/05}
\ausgabetermin{19.~Oktober 2004}
\handouttitle{F}
\nummer{1}
\thema{Thema}

\newcommand{\Coma}{\textsl{Coma}}


%\usepackage[english, german]{babel}

\usepackage{hevea}
\usepackage[latin1]{inputenc}

%\uebungfalse
%\handouttrue

%%% common macro files for the projects


\newcommand{\NONAME}{Somename}


\newcommand{\Coma}{\textsl{Coma}}

\newcommand{\cvs}{cvs}
\newcommand{\Java}{\textsc{Java}}
\newcommand{\Cplusplus}{C$^{++}$}
\newcommand{\javadoc}{\textsc{javadoc}}



\newcommand{\team}[1]{\textbf{Responsible:} #1\bigskip{}}


\newcommand{\bnfdef}               {::=}
\newcommand{\bnfbar}               {\mathrel{|}}

\newcommand{\suchthat}{\mathrel{\mid}}
\newcommand{\union}  {\cup}
\newcommand{\intersect}  {\cap}
\newcommand{\sizeof}[1]  {{\mid}{#1}{\mid}}
\newcommand{\sem}    [2]        {[\![#2]\!]_{#1}}      %semantics

\newenvironment{diagram}{\begin{displaymath}}{\end{displaymath}}


%%% Local Variables: 
%%% mode: latex
%%% TeX-master: t
%%% End: 

\begin{document}

%\ausgabe{19.~Oktober 2004}
%\maketitle{x}{Koordination mit CVS}


\thispagestyle{empty}


\section*{Verteiltes Arbeiten unter CVS}

Zu Koordination der einzelnen Gruppen ist die Verwendung von
``\textit{Concurrent Version Control''} (CVS) vorgeschrieben. Wir werden
eine kurze Einf�hrung in CVS geben und einen \textit{account} als Server
f"ur das \emph{Repositorium} zur Verf�gung stellen. Die wichtigesten Dinge
sind auf diesem Handout zusammengefa�t. \ifweb\else Weitere Informationen
�ber unsere Webseite. \fi

\subsection*{Technische Vorraussetzungen}

Man ben�tigt zwei Dinge, um von Ferne auf das gemeinsame Repositorium
zugreifen zu k�nnen:
\begin{enumerate}
\item \textsl{ssh} (secure shell) f"ur den sicheren Zugriff und
\item \textsl{cvs} + \textsl{rcs} als Versionskontrolle.
\end{enumerate}


Diese Tools sind auf dem Uni-Netz vorhanden. Wer daheim (oder von einem
sonstigen account) arbeiten will, mu"s daf"ur sorgen, da"s sie installiert
sind. Sie sind kostenlos f"ur Linux/Unix/Windows/MacOS verf"ugbar.



\subsubsection*{ssh-Zugriff}



F�r den ersten Punkt, dem Fernzugriff unter \textsl{ssh}, geht man wie
folgt vor.  Man beschaffe sich einen \textsl{ssh}-Schl�ssel Der Schl�ssel
liegt typischerweise unter \texttt{\home{}/.ssh/identity.pub}.  Diesen
maile man an \texttt{swprakt@informatik.uni-kiel.de}. Wer einen solchen
Schl�ssel nicht hat, mu� ihn sich generieren und zwar mittels
\textsl{ssh-keygen} am besten \texttt{ssh-keygen -tdsa} f�r einen ssh-2
Schl�ssel.


\subsubsection*{Angabe des cvs-Servers}

F�r das korrekte Funktionieren mu� man cvs noch mitteilen, wo sich das
Repositorium befindet. Dies kann man wie folgt erreichen (bash-Syntax):

\begin{verbatim}
    export CVSROOT=swprakt@goofy.informatik.uni-kiel.de:/home/swprakt/cvsroot 
    export CVS_RSH=ssh
    export CVSEDITOR=emacs

    export CLASSPATH=$WORKDIR/coma/src:
\end{verbatim}



wobei \texttt{\$WORKDIR} ein Platzhalter (!) f�r das Verzeichnis ist, in
dem man arbeiten wird. Die ersten drei Zeilen sind zum korrekten Steuern
von cvs gedacht.\footnote{Man kann cvs auch anders steuern, z.B. �ber
  .cvsrc, bei Bedarf bitte selbst im Manual nachschlagen oder nachfragen.}
Anstelle \textsl{emacs} kann man auch \textsl{emacsclient} oder den Editor
seiner Wahl nehmen.  \textsl{emacsclient} sollte man dann nehmen, wenn man
ohnehin emacs verwendet und beim starten des emacs' ein
\texttt{(server-start)} ausgef�hrt wird.  Die letzte Zeile hat nichts mit
ssh/cvs, sondern hat mit dem diesj"ahrigen Java-Projekt selbst zu tun.


\subsection*{Beispiel}


Das Beispiel verwendet \texttt{\$WORKDIR} als Bezeichnung f"ur das
\emph{Arbeitsverzeichnis} Als \texttt{\$WORKDIR} kann sich jeder selbst
passend w"ahlen, wo er das Arbeitsverzeichnis haben will. Es spricht nichts
dageben, da"s man sich auch mehrere Arbeitsverzeichnisse verschafft, zum
Beispiel eines auf seinem Uni-account und eines daheim, oder da"s ein Team
zwei Arbeitsverzeichnisse hat, an denen es getrennt arbeitet. Mit der Zahl
der ausgecheckten Arbeitsverzeichnisse steigt nat"urlich die M"oglichkeiten
der Verwirrung.  In der Regel kommt man mit einem Arbeitsverzeichnis pro
Person und pro ``Arbeitsplatz'' aus.
\begin{itemize}
\item \textbf{Auschecken:} So bekommt man zum ersten Mal seine Arbeitskopie
  des \Coma-Projektes:
\begin{verbatim}
              mkdir $WORKDIR
              cd $WORKDIR
              cvs checkout coma
\end{verbatim}
  Die Variable \texttt{\$WORKDIR} ist hier nur zur Illustration gew"ahlt,
  cvs kennt sie nicht. Falls man konkret \texttt{\$WORKDIR} =
  \texttt{\home{}/Projekt} w"ahlt, hei�en die Befehle:
\begin{verbatim}
              cd ~
              mkdir Projekt
              cd Projekt
              cvs checkout coma/src
\end{verbatim}
  Im Folgenden bezeichnet \texttt{\$WORKDIR} immer das frei w�hlbare
  Arbeitsverzeichnis. Macht man
\begin{verbatim}
              cvs checkout coma/org
\end{verbatim}
  bekommt man die Gruppeneinteilung. Macht man nur
\begin{verbatim}
              cvs checkout coma
\end{verbatim}
  so bekommt man alles zum \Coma-Projekt, einschlie�lich der Quellen des
  Pflichtenheftes und der Web-Seite.
\item \textbf{Zur�ckspeichern:} Wenn man mit den �nderungen durch ist, kann
  man den gesamten Verzeichnisbaum zur�ckspeichern:\footnote{Beachte den
    Wechsel in das Unterverzeichnis.}
\begin{verbatim}
              cd $WORKDIR/coma
              cvs commit 
\end{verbatim}
  Danach wird man aufgefordert, einen Kommentar bez�glich seiner �nderungen
  anzugeben.  Falls der Kommentar nur kurz ist, kann man schneller auch
\begin{verbatim}
              cvs commit -m"Anmerkungen zu den �nderungen"
\end{verbatim}
  verwenden. 
\item \textbf{Updaten:} Den Baum seines Arbeitsverzeichnisses auf den
  neuesten Stand bringen, geht so
\begin{verbatim}
               cd $WORKDIR/coma
               cvs update
\end{verbatim}
\item \textbf{Neue Dateien + Verzeichnisse:} Das geht mit
\begin{verbatim}
               cvs add [filename]
\end{verbatim}
  entfernen mit
\begin{verbatim}
               cvs remove [filename],
\end{verbatim}
  Anstelle eines Dateinamens kann man auch ein Verzeichnis hinzuf�gen oder
  auch alle neuen java-Dateien mit \texttt{cvs add *.java}.
\end{itemize}

\subsection*{Strategie und Spielregeln}

Versionskontrolle garantiert kein reibungsloses Arbeiten, es
unterst�tzt dies, wenn man gewisse Disziplin wahrt. Folgende Daumenregeln:

\begin{itemize}
\item Im Allgemeinen gilt: sobald die Gefahr besteht, da"s eine "Anderung
  die anderen Gruppen in Mitleidenschaft ziehen kann, soll dies in der
  Regel vorher abgekl"art werden. Insbesondere:
  \begin{itemize}
  \item �ndern von globalen Paketen/Paketen von anderen Gruppen: nur nach
    reiflicher �ber\-legung und R�ck\-sprache mit der betroffenen Gruppe.
  \item Keine unangek�ndigte �nderung der \emph{Verzeichnisstruktur} im
    Repositorium. Neue Unterverzeichnisse im eigenen Paket sind dabei in
    Ordnung. Man soll auch die Finger von den administrativen cvs-Dateien
    lassen.
  \item kein globales R"uckg"angigmachen von "Anderungen anderer Seite ohne
    R"ucksprache.
  \item Keine ``\emph{watches}'' setzen (au"ser eventuell auf seinen
    eigenene Code).
  \end{itemize}
\item Nur \emph{kompilierbare} Versionen einchecken, d.h.\ der eigene Teil
  mu� sich mittels \texttt{make all} ohne Fehlermeldungen kompilieren
  lassen. Dies gilt noch nicht f�r die Anlaufphase, bis die Pakete zum
  ersten Mal integriert werden. �nderungen, die die \emph{Schnittstellen}
  mit einem anderen Paket betreffen sollten \emph{angek�ndigt} werden, zum
  Beispiel in den Besprechungen.  Daneben sind die Mailadressen der
  einzelnen Gruppen und ihrer Teilnehmer \ifweb\url{org/gruppen.txt}{hier}
  zugreifbar \else im Netz vorhanden\fi.
\item \texttt{Makefile}s und \texttt{Readme}s sind hilfreich. Der Code f"ur
  jedes Pakets soll mittels \texttt{make all} als erstes ``target'' des
  Makefiles kompilierbar sein. Im Paket \texttt{absynt} finden sich
  Beispiele.
\item Der Code soll sinnvoll mittels \textsl{Javadoc} kommentieren werden.
  Zumindest der Author/die Autoren sollen dokumentiert sein.  In Paket
  \texttt{absynt} kann man sich anschauen, wie man auch cvs-logs
  dokumentieren lassen kann, (z.B.\ indem man die Datei
  \texttt{templates/classtemplate.txt} verwendet und anpa�t.) Die
  generierte Dokumentation wird in regelm"a"sigen Abst"anden im Netz
  bereitgestellt.
\end{itemize}



\nocite{cvs:www}
\nocite{cvs:manual}
\nocite{fogel:cvs}

\bibliographystyle{alpha}

\addcontentsline{toc}{section}{Bibliography}


{%\small
  \bibliography{string,etc,imperative,oop,types,protocol,functional,logic,crossref}
}

\end{document}



%%%%%%%%%%%%%%%%%%%%%%%%%%%%%%%%%%%%%%%%%%%%%%%%%%%%%%%%%%%%%
%% $Id: handout1.tex,v 1.2 2004/10/19 07:59:22 Steffen Exp $ 
%%%%%%%%%%%%%%%%%%%%%%%%%%%%%%%%%%%%%%%%%%%%%%%%%%%%%%%%%%%%%


%%% Local Variables: 
%%% mode: latex
%%% TeX-master: t
%%% End: 

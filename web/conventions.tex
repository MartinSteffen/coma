


\section*{Conventions and rules of the game}
\label{sec:conventions}

Besides the \emph{CVS-strategies} which have been handed out and discussed,
we should take care of the following:

\begin{itemize}
\item \textbf{Makefile}: each package, i.e., the root of the respective
  sub-directory must containt a \texttt{Makefile}.  Its first target must
  be \texttt{make all}, which generates for the packges the \Java{} byte
  code (whatever it takes in intermediate steps).  Additionally, a target
  \texttt{make clean} must be supported, which removes the byte-code and
  all temporary files again.  An example for a workable Makefile is
  contained in the \texttt{absynt}-package which covers the (\emph{very
    simple}) needs just described.

  
\item \textbf{Documentation:} the \Java-code should be meaningfully
  commented and documented. Information about the implementations which are
  relevant for the co-developers, are done in \javadoc. This especially
  applies to the intended meaning and usage of the public interfaces
  between the packages. Part of the documentation should be the name of the
  developer(s) and the version of the file. Further useful information
  could be the status of the method, class, \ldots, e.g., whether the
  implementation currently is only available as stub, whether the
  functionality is finished, but not yet tested, which are the known
  bugs/features \ldots.
  
  The \javadoc-comments may serve the internal communication to state what
  the package offers (and what it assumes). The generated web-pages are
  available via the project page, they are refreshed from time to time, and
  kept
  \url{http://www.informatik.uni-kiel.de/inf/deRoever/SS02/Java/Slime/doc}{here}.
\end{itemize}



%%%%%%%%%%%%%%%%%%%%%%%%%%%%%%%%%%%%%%%%%%%%%%%%%%%%%%%%%%%%
%% $Id: conventions.tex,v 1.1 2004/10/18 06:38:33 ms Exp $
%%%%%%%%%%%%%%%%%%%%%%%%%%%%%%%%%%%%%%%%%%%%%%%%%%%%%%%%%%%%
%%% Local Variables: 
%%% mode: latex
%%% TeX-master: "main"
%%% End: 

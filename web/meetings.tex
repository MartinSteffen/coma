
\subsection*{Weekly meetings}
\label{sec:meetings}


This section contains the results of the weekly meetings as communicated
via email to the group members. There are kept here for quick reference.




  \begin{table}[htbp]
    \centering
    \begin{tabular}[t]{r@{\quad}l@{\quad\quad}p{9cm}}
    \\\hline
    0.
    &
    \href{meetings/2004-10-19.txt}{19. October}
    &
    kick-off
    \\
    1.
    &
    \href{meetings/2004-10-26.txt}{26. October}
    &
    general jamboree
    \\
    2.
    &
    \href{meetings/2004-11-02.txt}{2. November}
    &
    Tools discussion
    \\
    3.
    &
    \href{meetings/2004-11-09.txt}{9. November}
    &
    Spec. presentations
    \\
    4.
    &
    \href{meetings/2004-11-15-a.txt}{announcement (15.11)},
    \href{meetings/2004-11-16.txt}{16. November}
    &
    group forming/next steps
    \\
    5.
    &
    \href{meetings/2004-11-23.txt}{23. November}
    &
    Various topics, in particular: status
    \\
    5.
    &
    \href{meetings/2004-11-30.txt}{30. November}
    &
    \textbf{SQL shootout}, communicaton problems,
    (\href{figures/db_photo.jpg}{blackboard} for reference)
    \\
    6.
    &
    \href{meetings/2004-12-07.txt}{7. December}
    &
    status, various topics, progress?
    \\\hline
    \iffalse
    1.
    &
    \url{meetings/meeting-160402.txt}{16. April}
    &
    task assignment, change of meeting time
    \\
    2.
    & 
    \url{meetings/meeting-240402.txt}{24. April}
    &
    no real decisions, preparation for next meeting
    \\
    3.
    &
    1.\ May
    &
    holiday
    \\
    4.
    &
    \url{meetings/meeting-080502.txt}{8.\ May}
    & 
    task presentation, status of editor group unclear,
    check group said ciao, restructuring planned
    \\
    5.
    &
    15.\ May
    &
    no email?
    \\
    6.
    &
    22. May
    &
    no email?
    \\
    7.
    &
    29.\ May
    &
    no email
    \\
    8.
    &
    \url{meetings/meeting-050602.txt}{5.\ June}
    &
    makeshift parser under utilites added; first
    integration set to 12.\ June
    \\
    9.
    &
    \url{meetings/meeting-120602.txt}{12.\ June}
    &
    restructuring now (freeze)!, makeshift checks will be implemented 
    (as visitor or otherwise), no integration (since code
    missing)
    \\
    10. 
    &
    \url{meetings/meeting-190602.txt}{19.\ June}
    &
    no integration yet
    \\
    11. 
    &
    \url{meetings/meeting-260602.txt}{26.\ June}
    &
    everyone on board; plan of final review;
    plan for last 3 weeks
    \\
    12. 
    &
    \url{meetings/meeting-030702.txt}{3.\ July}
    &
    removal of additional checked in stuff +
    removal of class files.
    \\
    13. 
    &
    \url{meetings/meeting-100702.txt}{10.\ July}
    &
    decisions about interface inconsistencies,
    making it compilable, 
    preparing the integration
    \\
    14. 
    &
    %\url{meetings/meeting-190602.txt}{17.\ July}
    17.\ July
    &
    no meeting
    \\
\fi
  \end{tabular}
    \caption{Meeting minutes}
    \label{tab:meetings}
  \end{table}



\begin{tabular}{|l|l|l|}
\hline
week & date & topic 
\\\hline
01 & 19.10. & Introduction\\
02 & 26.10. & Discussion\\
03 & 02.11. & Tools \& Architecture\\
04 & 09.11. & Specification\\
05 & 16.11. & {Milestone 0}: Specification / Tools (\LaTeX\ source)\\
06 & 23.11. &\\
07 & 30.11. &  {Milestone 1:} ...\\
08 & 07.12. &\\
09 & 14.12. &\\
10 & 21.12. & Milestone 2: ...\\
11 & 11.01. &\\
12 & 18.01. &\\
13 & 25.01. &\\
14 & 01.02. & Milestone 3: ...\\
15 & 08.02. & Presentation of final product\\
\hline
\end{tabular}



%Organisatorial/procedural things discussed during the meeting at 3.7.02
%included arguments mentioned \url{slides/meeting030702.ps}{here.}



\iffalse
Here the \important{official decision} concerning the \texttt{CLASSPATH}
etc. (same is in \texttt{Readme.devel}).

\begin{itemize}
\item the \importantxx{checked-in} versions of cup and lex are
  obligatory\footnote{They replace the ones previously used under
    \texttt{/home/java}, which had been the official ones so far. In
    effect, they are the same, and just checked in.}
\item the following \importantxx{classpath} is obligatory:
  \texttt{CLASSPATH=<WORKDIR>/Slime/src:<WORKDIR>/Slime/lib/JLex.jar:<WORKDIR>/Slime/lib/java_cup.jar:}
  where of course the \texttt{<WORKDIR>} is a placeholder and it has to be
  adapted by the individual user to his work directory.
\item no \importantxx{generated} files nor \importantxx{class} files will be
  checked in (unless technical reasons call for it, in which case we will
  discuss this in the light of the new arguments again)
\item no \importantxx{non-slime} code or data will be checked in under
  \texttt{CLASSPATH=<WORKDIR>/Slime/src}
\item revisions \emph{log}s are useful and worth reading, but it's not
  mandatory to read each other's logs.
\end{itemize}
\fi


%%% Local Variables: 
%%% mode: latex
%%% TeX-master: "main"
%%% End: 


\section*{Material}

Some material (handouts, slides) which have been presented during the
course).

\begin{tabular}{r@{\quad\quad}rll}
  \hline
  1. & 19. October & coma  & 
  \href{slides/main-coma.pdf}{slides}
  \\
  2. & 19. October & cvs (obsolete) & \href{handouts/handout1.pdf}{handout 1}
  \\
  3. & 9. November & subversion + some tool remarks &
  \href{handouts/handout2.pdf}{handout 2}
  \\
   & 16. November & bugzilla &
  \href{handouts/handout3.pdf}{handout 3}
  \\
   & 15. November & groups &
  \href{handouts/handout4.pdf}{organizational form}
  \\
   & 15. November & groups and next steps &
  \href{handouts/handout5.pdf}{handout 5}
  \\\hline 
   & 18. November &  &
  \href{spec/v2/main.pdf}{data spec} (v2)
  \\\hline
\end{tabular}




\subsection*{Spec + tools groups}
\label{sec:material.spec-tools}%

The following is the material worked out in the fist phase. It is also
checked in under \texttt{spec/material} in the main trunk of the
development.


\begin{tabular}{llll}
  14.11 & Spec 1 & 
  \href{spec/material/spec1/spezifikation.pdf}{specification}, 
  \href{spec/material/spec1/slides.pdf}{coma slides}, 
  & (in German)
  \\
  16.11 & Spec 2 & 
  \href{spec/material/spec2/usecase.pdf}{usecase}, 
  \href{spec/material/spec2/datenstruktur.ps}{Datenstruktur}, 
  \href{spec/material/spec2/slides.pdf}{coma slides}
  & (in German)
  \\
  10.11 & Alexander & \href{spec/material/derenbach/tools.pdf}{choice of tools}
  \\
  ?  & Tools 1 & ????
  \\
  15.11.04 & Tools 2 &
  \href{spec/material/tools2/tools.pdf}{choice of tools}
  \\
  16.11 & Tools 3 & \href{spec/material/tools3/tools.pdf}{choice of tools}
  \\
  - & Tests & \href{spec/material/tests/tests.txt}{Test concept}
\end{tabular}




\subsubsection*{Other stuff}

\begin{tabular}{lp{8cm}}
 \LaTeX\ \href{misc/handout.cls}{handout class}  +
 \href{misc/hevea.sty}{hevea} & for tex'ing the handouts, if wished.
 \\
 \href{http://snert.informatik.uni-kiel.de:8080/~swprakt/phpBB2/}{bulletin board}  
 \\
 \href{http://snert.informatik.uni-kiel.de:8080/~swprakt/bugzilla}{bugzilla} 
 &
 login via email-address + your chosen password (not the generated one, not the ssh-passphrase)
\end{tabular}





%%% Local Variables: 
%%% mode: latex
%%% TeX-master: "main"
%%% End: 

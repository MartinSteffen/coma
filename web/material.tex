
\section*{Sources of information}


Some material (handouts, slides) which have been presented during the
course) or circulated via the email list.



\begin{tabular}{r@{\quad\quad}rll}
  \hline
  1. & 19. October & coma  & 
  \href{slides/main-coma.pdf}{slides}
  \\
  2. & 19. October & cvs (obsolete) & \href{handouts/handout1.pdf}{handout 1}
  \\
  3. & 9. November & subversion + some tool remarks &
  \href{handouts/handout2.pdf}{handout 2}
  \\
   & 16. November & bugzilla &
  \href{handouts/handout3.pdf}{handout 3}
  \\
   & 15. November & groups &
  \href{handouts/handout4.pdf}{organizational form}
  \\
   & 15. November & groups and next steps &
  \href{handouts/handout5.pdf}{handout 5}
  \\\hline 
   & 18. November &  &
  \href{spec/v2/main.pdf}{data spec} (v2)
  \\\hline 
   & 30. November &  &
  \href{handouts/handout6.pdf}{communication problems}
  \\
   & 31. Januar &  &
  \href{handouts/handout7.pdf}{E-O-S demo}
  \\
\end{tabular}




\begin{description}
\item[BB]
  \href{http://snert.informatik.uni-kiel.de:8080/~swprakt/phpBB2/}{bulletin
    board}: not obligatory, but perhaps useful.
\item[Bugzilla] 
  \href{http://snert.informatik.uni-kiel.de:8080/~swprakt/bugzilla}{bugzilla}: 
  login via email-address + your chosen password (not the generated one, not the ssh-passphrase)
  For some hints, see also handout 3.
\item[Repository:] The repository resides on the \texttt{snert}-machine.
  Write-access is password-protected, for browsing the current status, you
  can click here
  \href{http://snert.informatik.uni-kiel.de:8080/svn/coma/}{\textbf{current
      repository status}}. Write access is granular in the following sense:
  each group can write only within its own domain of responsibilities.
\item[examples:] some examples you can use for inspiration
  \begin{itemize}
  \item a \href{requirements/examples/cfpapers.html}{ ``call-for-papers''}
    to have a look how a conference advertisement for authors could look
    like.
  \item an empty \href{requirements/examples/reviewform.txt}{ascii review
      form} for some conference
  \end{itemize}
\end{description}





\subsection*{Weekly meetings}
\label{sec:meetings}


This section contains the results of the weekly meetings as communicated
via email to the group members. There are kept here for quick reference.




  \begin{table}[htbp]
    \centering
    \begin{tabular}[t]{r@{\quad}l@{\quad\quad}p{9cm}}
    \\\hline
    0.
    &
    \href{meetings/2004-10-19.txt}{19. October}
    &
    kick-off
    \\
    1.
    &
    \href{meetings/2004-10-26.txt}{26. October}
    &
    general jamboree
    \\
    2.
    &
    \href{meetings/2004-11-02.txt}{2. November}
    &
    Tools discussion
    \\
    3.
    &
    \href{meetings/2004-11-09.txt}{9. November}
    &
    Spec. presentations
    \\
    4.
    &
    \href{meetings/2004-11-15-a.txt}{announcement (15.11)},
    \href{meetings/2004-11-16.txt}{16. November}
    &
    group forming/next steps
    \\
    5.
    &
    \href{meetings/2004-11-23.txt}{23. November}
    &
    Various topics, in particular: status
    \\
    5.
    &
    \href{meetings/2004-11-30.txt}{30. November}
    &
    \textbf{SQL shootout}, communicaton problems,
    (\href{figures/db_photo.jpg}{blackboard} for reference)
    \\
    6.
    &
    \href{meetings/2004-12-07.txt}{7. December}
    &
    status, various topics, progress?
    \\
    7.
    &
    \href{meetings/2004-12-14.txt}{14. December}
    &
    status/progress, SQL, \textbf{testing}, svn
    \\
    8.
    &
    \href{meetings/2004-12-21.txt}{21. December}
    &
    status/progress, nothing particular
    \\\hline
    9.
    &
    \href{meetings/2005-01-11.txt}{11. January}
    &
    [status/progress], testing, DB problems
    \\
    10.
    &
    \href{meetings/2005-01-18.txt}{18. January}
    &
    status/progress report, plan for the rest
    \\
    11.
    &
    \href{meetings/2005-01-25.txt}{25. January}
    &
    nothing specific, date for end-demo
    \\
    12.
    &
    \href{meetings/2005-02-01.txt}{1. February}
    &
    last orders, handout 7
    \\
    12.
    &
    \href{meetings/2005-02-08.txt}{8. February}
    &
    ``\textbf{E-O-S}''
    \\\hline
    \iffalse
    1.
    &
    \url{meetings/meeting-160402.txt}{16. April}
    &
    task assignment, change of meeting time
    \\
    2.
    & 
    \url{meetings/meeting-240402.txt}{24. April}
    &
    no real decisions, preparation for next meeting
    \\
    3.
    &
    1.\ May
    &
    holiday
    \\
    4.
    &
    \url{meetings/meeting-080502.txt}{8.\ May}
    & 
    task presentation, status of editor group unclear,
    check group said ciao, restructuring planned
    \\
    5.
    &
    15.\ May
    &
    no email?
    \\
    6.
    &
    22. May
    &
    no email?
    \\
    7.
    &
    29.\ May
    &
    no email
    \\
    8.
    &
    \url{meetings/meeting-050602.txt}{5.\ June}
    &
    makeshift parser under utilites added; first
    integration set to 12.\ June
    \\
    9.
    &
    \url{meetings/meeting-120602.txt}{12.\ June}
    &
    restructuring now (freeze)!, makeshift checks will be implemented 
    (as visitor or otherwise), no integration (since code
    missing)
    \\
    10. 
    &
    \url{meetings/meeting-190602.txt}{19.\ June}
    &
    no integration yet
    \\
    11. 
    &
    \url{meetings/meeting-260602.txt}{26.\ June}
    &
    everyone on board; plan of final review;
    plan for last 3 weeks
    \\
    12. 
    &
    \url{meetings/meeting-030702.txt}{3.\ July}
    &
    removal of additional checked in stuff +
    removal of class files.
    \\
    13. 
    &
    \url{meetings/meeting-100702.txt}{10.\ July}
    &
    decisions about interface inconsistencies,
    making it compilable, 
    preparing the integration
    \\
    14. 
    &
    %\url{meetings/meeting-190602.txt}{17.\ July}
    17.\ July
    &
    no meeting
    \\
\fi
  \end{tabular}
    \caption{Meeting minutes}
    \label{tab:meetings}
  \end{table}



\begin{tabular}{|l|l|l|}
\hline
week & date & topic 
\\\hline
01 & 19.10. & Introduction\\
02 & 26.10. & Discussion\\
03 & 02.11. & Tools \& Architecture\\
04 & 09.11. & Specification\\
05 & 16.11. & {Milestone 0}: Specification / Tools (\LaTeX\ source)\\
06 & 23.11. &\\
07 & 30.11. &  {Milestone 1:} ...\\
08 & 07.12. &\\
09 & 14.12. &\\
10 & 21.12. & Milestone 2: ...\\
11 & 11.01. &\\
12 & 18.01. &\\
13 & 25.01. &\\
14 & 01.02. & Milestone 3: ...\\
15 & 08.02. & Presentation of final product\\
\hline
\end{tabular}



%Organisatorial/procedural things discussed during the meeting at 3.7.02
%included arguments mentioned \url{slides/meeting030702.ps}{here.}



\iffalse
Here the \important{official decision} concerning the \texttt{CLASSPATH}
etc. (same is in \texttt{Readme.devel}).

\begin{itemize}
\item the \importantxx{checked-in} versions of cup and lex are
  obligatory\footnote{They replace the ones previously used under
    \texttt{/home/java}, which had been the official ones so far. In
    effect, they are the same, and just checked in.}
\item the following \importantxx{classpath} is obligatory:
  \texttt{CLASSPATH=<WORKDIR>/Slime/src:<WORKDIR>/Slime/lib/JLex.jar:<WORKDIR>/Slime/lib/java_cup.jar:}
  where of course the \texttt{<WORKDIR>} is a placeholder and it has to be
  adapted by the individual user to his work directory.
\item no \importantxx{generated} files nor \importantxx{class} files will be
  checked in (unless technical reasons call for it, in which case we will
  discuss this in the light of the new arguments again)
\item no \importantxx{non-slime} code or data will be checked in under
  \texttt{CLASSPATH=<WORKDIR>/Slime/src}
\item revisions \emph{log}s are useful and worth reading, but it's not
  mandatory to read each other's logs.
\end{itemize}
\fi


%%% Local Variables: 
%%% mode: latex
%%% TeX-master: "main"
%%% End: 






\subsection*{Spec + tools groups}
\label{sec:material.spec-tools}%

The following is the material worked out in the first phase. It is also
checked in under \texttt{spec/material} in the main trunk of the
development.


\begin{tabular}{llll}
  14.11 & Spec 1 & 
  \href{spec/material/spec1/spezifikation.pdf}{specification}, 
  \href{spec/material/spec1/slides.pdf}{coma slides}, 
  & (in German)
  \\
  16.11 & Spec 2 & 
  \href{spec/material/spec2/usecase.pdf}{usecase}, 
  \href{spec/material/spec2/datenstruktur.ps}{Datenstruktur}, 
  \href{spec/material/spec2/slides.pdf}{coma slides}
  & (in German)
  \\
  10.11 & Alexander & \href{spec/material/derenbach/tools.pdf}{choice of tools}
  \\
  ?  & Tools 1 & \href{spec/material/tools1/tools.txt}{choice of tools}
  \\
  15.11.04 & Tools 2 &
  \href{spec/material/tools2/tools.pdf}{choice of tools}
  \\
  16.11 & Tools 3 & \href{spec/material/tools3/tools.pdf}{choice of tools}
  \\
  - & Tests & \href{spec/material/tests/tests.txt}{Test concept}
\end{tabular}




\subsubsection*{Other stuff}

\begin{tabular}[t]{lp{8cm}}
 \LaTeX\ \href{misc/handout.cls}{handout class}  (+
 \href{misc/hevea.sty}{hevea}) & for tex'ing the handouts, if wished. hevea is only needed if 
 you want to generate webpages out of it, i.e., if the switch \textrm{\webtrue} is set
\end{tabular}


\subsection*{Tools potpourri \& matrix}

The following gives a short overview over the tools, languages \ldots
needed for the project. The are classified according to the differnt
developer grouds and the targeted users.

Entries $+$ mean ``available'' or ``required'', $-$ means ``not available''
or ``not required'', and $o$ means ``unclear''. Entries in [brackets] means
``not fully available'' or ``almost not available'', or similar.

In most cases, the \emph{versions} are those from
\texttt{snert.}\footnote{Note that for the various java lib's there's a
  little confusions concerning the numbers (and file names) as some of the
  jar files which are offered for download at
  \href{http://snert.informatik.uni-kiel.de:8080/~wprguest3/downloads/}{the
    download space of the java group} are just duplicates (with the same
  version number, sometimes different names??) of the jars, which are
  installed \emph{anyway} on snert as part of tomcat. One should base the
  tool on the \emph{official} ones, as part of the chosen tomcat
  installation. Everything else invites trouble.} Sometimes the tool
versions installed at the computer lab (or at private machines) differs,
but without much (or known) problems; even so: ultimately the
\emph{reference is snert} and there, globally installed stuff, not
something copied at some obscure private directory.



\begin{table}[htbp]
  \centering
  \small
  \begin{tabular}[t]{|lll|cc|ccccccc|}
    \hline
    name & version & description & \multicolumn{2}{|c|}{installed at} & \multicolumn{7}{|c|}{needed by}
    \\
    &
    &
    &
    &
    &
    \multicolumn{5}{|c}{developers}
    &
    \multicolumn{2}{c|}{customers}
    \\
    & &
    &
    snert
    &
    lab
    &
    p$_1$
    &
    p$_2$
    &
    j
    &
    test
    &
    org
    &
    server% side
    &
    client% side
    \\\hline%%%%
    \multicolumn{12}{|c|}{common development}
    \\\hline
    \href{http://ant.apache.org/}{ant} 
    &
    1.6.2 %% version
    &
    build tool % description
    & 
    +  % snert
    & 
    +  % lab
    &
    o %p1
    &
    o %p2
    &
    + %j
    &
    [+] % t
    & 
    o %o
    &
    + % installer
    &
    - % external user
    \\
    \href{http://www.gnu.org/software/make}{gnu make} 
    &
    3.80 %% version
    &
    build tool % description
    & 
    +  % snert
    & 
    +  % lab
    &
    o %p1
    &
    o %p2
    &
    - %j
    &
    o % t
    & 
    + %o
    &
    + % installer
    &
    - % external user
    \\
    \href{http://subversion.tigris.org/}{subversion}
    &
    1.0.9 %% version
    & 
    CM % description
    & 
    +  % snert
    & 
    [+]  % lab
    &
    + %p1
    &
    + %p2
    &
    + %j
    &
    + % t
    & 
    + %o
    &
    o % installer
    &
    - % external user
    \\\hline
    \multicolumn{12}{|c|}{common basis}
    \\\hline
    \href{http://www.apache.org}{apache}
    &
    2.0.51 %% version
    & 
    web server % description
    & 
    +  % snert
    & 
    [-]  % lab
    &
    + %p1
    &
    + %p2
    &
    + %j
    &
    + % t
    & 
    - %o
    &
    + % server side
    &
    - % client side
    \\
    \href{http://www.mysql.org}{mysql}
    &
    3.23.58 %% version
    & 
    data base % description
    & 
    +  % snert
    & 
    [-]  % lab
    &
    + %p1
    &
    + %p2
    &
    + %j
    &
    + % t
    & 
    - %o
    &
    + % server side
    &
    - % client side
    \\\hline
    \multicolumn{12}{|c|}{``java''}
    \\\hline
    \href{http://jakarta.apache.org/tomcat/index.html}{jakarta tomcat}
    &
    5.0.30 %% version
    & 
    web server % description
    & 
    +  % snert
    & 
    -  % lab
    &
    - %p1
    &
    - %p2
    &
    + %j
    &
    + % t
    & 
    - %o
    &
    + % server side
    &
    - % client side
    \\
    \href{http://java.sun.com/j2se/index.jsp}{java}
    &
    1.5.0 %% version
    & 
    language % description
    & 
    +  % snert
    & 
    +  % lab
    &
    - %p1
    &
    - %p2
    &
    + %j
    &
    + % t
    & 
    - %o
    &
    + % server side
    &
    - % client side
    \\
    \href{http://www.informatik.uni-kiel.de/~java}{java lib's}
    &
    various %% version
    & 
    further libs % description
    & 
    [+] % snert
    & 
    [+]  % lab
    &
    - %p1
    &
    - %p2
    &
    + %j
    &
    + % t
    & 
    - %o
    &
    + % server side
    &
    - % client side
    \\\hline
    \multicolumn{12}{|c|}{``php''}
    \\\hline
    \href{https://www.php.net}{php}
    &
    4.3.8 %% version
    & 
    scripting lang. % description
    & 
    +  % snert
    & 
    -  % lab
    &
    + %p1
    &
    + %p2
    &
    - %j
    &
    + % t
    & 
    - %o
    &
    + % server side
    &
    - % client side
    \\
    \href{http://www.phpdoc.org}{php doc}
    &
    %% version
    & 
    doc generation
    & 
    +  % snert
    & 
    -  % lab
    &
    o %p1
    &
    o %p2
    &
    - %j
    &
    o % t
    & 
    - %o
    &
    - % server side
    &
    - % client side
    \\\hline
    \multicolumn{12}{|c|}{common testing}
    \\\hline
    \href{http://www.junit.org}{junit}
    &
    %% version
    &
    unit testing
    & 
    +  % snert
    & 
    +  % lab
    &
    - %p1
    &
    - %p2
    &
    + %j
    &
    + % t
    & 
    - %o
    &
    - % server side
    &
    - % client side
    \\
    \href{http://www.sebastian-bergmann.de/en/phpunit.php}{php unit} 
    &
    &
    unit testing
    & 
    +  % snert
    & 
    +  % lab
    &
    [+] %p1
    &
    [+] %p2
    &
    - %j
    &
    + % t
    & 
    - %o
    &
    - % server side
    &
    - % client side
    \\\hline
    \multicolumn{12}{|c|}{common spec + doc}
    \\\hline
    \href{https://www.esm.jp/jude-web/en/index.html}{jude}
    &
    %% version
    & 
    design % description
    & 
    +  % snert
    & 
    +  % lab
    &
    + %p1
    &
    + %p2
    &
    + %j
    &
    + % t
    & 
    + %o
    &
    - % server side
    &
    - % client side
    \\
    \href{http://www.latex-project.org/}{\LaTeX}
    &
    tetex 2.0.2 %% version
    & 
    doc % description
    & 
    +  % snert
    & 
    + % lab
    &
    o %p1
    &
    o %p2
    &
    o %j
    &
    o % t
    & 
    + %o
    &
    o % server side
    &
    - % client side
    \\
    \href{http://pauillac.inria.fr/~maranget/hevea}{hevea}
    &
    1.07 %% version
    & 
    doc/html 
    & 
    - % snert
    & 
    + % lab
    &
    o %p1
    &
    o %p2
    &
    o %j
    &
    o % t
    & 
    + %o
    &
    - % server side
    &
    - % client side
    \\\hline
  \end{tabular}
  \caption{Tools, software, etc}
  \label{tab:tools}
\end{table}

\subsection*{External links}

\begin{itemize}
\item \href{http://java.sun.com/j2se/index.jsp}{J2SE}
\item \href{http://java.sun.com/j2ee/index.jsp}{J2EE}
\item \href{http://java.sun.com/products/ejb/}{EJB}
\item \href{http://java.sun.com/products/jsp/index.jsp}{JSP}
\item \href{http://java.sun.com/products/servlet/index.jsp}{Servelets}
\item \href{http://www.mysql.com/products/connector/j/}{JDBC}
\item \href{http://www.jboss.org/}{JBoss App. Server}
\item \href{http://subversion.tigris.org/}{Subversion}
\item \href{http://www.phpdoc.org/docs/HTMLSmartyConverter/default/phpDocumentor/tutorial_tags.pkg.html}{PhP DOC}
\item \href{http://www.pushtotest.com/}{Push2Test}
\end{itemize}




\subsection*{Alternatives}


\iffalse

        <li> 
        <li> : 500\$ the licensed version
        <li> <a href="http://borbala.com/cyberchair/">Cyberchair</a>: unknown price
        <li> <a href="http://www.ifi.uio.no/confman/ABOUT-ConfMan/">Confman</a>: Probably not (yet) very stable
        <li> <a href="http://store.lantrax.com/lantraxinc/mecoma.html">Metaframe:</a> seems to cost lots of money
      </ul>
\fi
Here's a list of available conference managers of comparable intentions.
\begin{description}
\item[\href{http://www.softconf.com/START/}{start}:] widely used
\item[\href{http://borbala.com/cyberchair/}{CyberChair:}] some well-known
  conference manager, in a free and a (more modern, more stable etc)
  licensed version. It's based on \mathit{python}
\item[\href{http://www.zakongroup.com/technology/openconf.shtml}{openconf}:]
\item[\href{http://www.paperdyne.de}{Paperdyne}:] another commercial thing
  nice demo
\end{description}



%%% Local Variables: 
%%% mode: latex
%%% TeX-master: "main"
%%% End: 

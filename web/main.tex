


\newif\ifweb\webtrue
\def\version{5}
\def\status{transient}
\def\docdate{17.~01.~2005}




%%%%%%%%%%%%%%%%%%%%%%%%%%%%%%%%%%%%%%%%%%%%%%%%%%%%%%%%%%%%%%%
%% $Id$ %%
%%%%%%%%%%%%%%%%%%%%%%%%%%%%%%%%%%%%%%%%%%%%%%%%%%%%%%%%%%%%%%%
%%% Local Variables: 
%%% mode: latex
%%% TeX-master: "main"
%%% End: 


\documentclass[11pt]{article}
\usepackage{hevea}
\usepackage{hyperref}
%\usepackage{url}
\usepackage[english]{babel}
\usepackage{epsfig}
\usepackage{tfheader}
\usepackage{a4wide}
\usepackage[latin1]{inputenc}

%% common macro files for the projects


\newcommand{\NONAME}{Somename}


\newcommand{\Coma}{\textsl{Coma}}

\newcommand{\cvs}{cvs}
\newcommand{\Java}{\textsc{Java}}
\newcommand{\Cplusplus}{C$^{++}$}
\newcommand{\javadoc}{\textsc{javadoc}}



\newcommand{\team}[1]{\textbf{Responsible:} #1\bigskip{}}


\newcommand{\bnfdef}               {::=}
\newcommand{\bnfbar}               {\mathrel{|}}

\newcommand{\suchthat}{\mathrel{\mid}}
\newcommand{\union}  {\cup}
\newcommand{\intersect}  {\cap}
\newcommand{\sizeof}[1]  {{\mid}{#1}{\mid}}
\newcommand{\sem}    [2]        {[\![#2]\!]_{#1}}      %semantics

\newenvironment{diagram}{\begin{displaymath}}{\end{displaymath}}


%%% Local Variables: 
%%% mode: latex
%%% TeX-master: t
%%% End: 



\title{{\huge\bf Coma} (under constant construction)}
%\author{}
%\url{}{Martin Steffen}}
\date{}



\ifweb

\htmlfoot{\hrulefill{}
  {\footnotesize Pages last (re-)generated \today}}
\renewcommand{\@bodyargs}{bgcolor="white" alink="red" vlink="\#407999"  link="\#7070ff"}  
\fi




\begin{document}
\vspace{-2cm}


%\pagestyle{empty}

\begin{rawhtml}

<a href="http://www.uni-kiel.de/">
  <img border=0 alt="[Christian-Albrechts-Universit�t]" height=30  src="http://www.uni-kiel.de/img/cau_logo.gif" align="left"></a>


<a href="http://www.tf.uni-kiel.de/">
  <img border=0 alt="[Technical Faculty]" height=30  src="/images/tflogo.gif" align="right"></a>
<br>

\end{rawhtml}



\maketitle{}


\begin{abstract}
  The is the \emph{organisatorial} overview page concerning this semester's
  project.  The ``official''
  \href{http://www.informatik.uni-kiel.de/inf/deRoever/WS0405/PITM}{web-page},
  in contrast, is more concerned with project-independant information and
  will not change much during semester.

\end{abstract}




%\tableofcontents



%\input{slime}
%
\section*{Review}
\label{sec:review}

There will be a final review, demo, presentation, and debriefing at the end of
the semester.


\iffalse
Date of the final review, demo, and debriefing is fixed to
\begin{center}
  \importantxx{\textbf{18.07.2002, 17:00}} 
\end{center}



\begin{table}[htbp]
  \centering
  \begin{tabular}{ll}
    Karsten Stahl, Martin Steffen & \url{review/overview.pdf}{overview}
    \\
    Andreas Niemann & \url{review/editor.pdf}{editor}
    \\
    Marco Wendel & \url{review/parser.pdf}{parser}
    \\
    Immo Grabe & simulator
    \\
    Karsten Stahl, Martin Steffen & \url{review/checks.pdf}{checks}
  \end{tabular}
  \caption{Presentations}
  \label{tab:presentations}
\end{table}

\fi



%%%%%%%%%%%%%%%%%%%%%%%%%%%%%%%%%%%%%%%%%%%%%%%%%%%%%%
%% $Id: review.tex,v 1.1 2004/10/18 06:38:33 ms Exp $
%%%%%%%%%%%%%%%%%%%%%%%%%%%%%%%%%%%%%%%%%%%%%%%%%%%%%%
%%% Local Variables: 
%%% mode: latex
%%% TeX-master: "main"
%%% End: 



%\section*{Java-doc}
%\label{sec:javadoc}
%\label{sec:html-doc}
%For convenient lookup for those who like \javadoc, the public documentation
%and comments are avaiable here. The pages are refreshed from time to time,
%especially after larger development steps.
%\begin{center}
%  \importantxx{documentation \url{doc}{Slime}}: currently (\today): last-minute
%  ``integrated'' tool
%\end{center}

%\section*{Snapshots}
\label{sec:snapshots}


The \emph{baselines} or \emph{snapshots} are archived here for quick
reference. %


\iffalse as \texttt{slime\_v[x].jar}, where \texttt{[x]} is the number or the
tag of the snapshot. Developers with access to the source repository can
retrieve those snapshots of course with cvs.\footnote{The command is
  \texttt{cvs update -r [tag]}. Be careful with this command, don't generate
  development branches without knowing.}


To \emph{execute} one of the archived snapshots, save it at some convenient
place, set the \texttt{CLASSPATH} (or use the \texttt{-classpath} option)
of \texttt{java}) to include the saved jar-file + the required versions of
\texttt{JLex} and \texttt{java\_cup}, and do\footnote{Since the archive is
  not stand-alone in that it requires additionally at run-time the lexer
  and the parser generator, the archive is not executable with the
  \texttt{-jar} option.}

\begin{center}
  \texttt{java <slimeversion>.jar slime.Main}
\end{center}


The \emph{required} version is \texttt{jdk1.4,} earlier version may cause
trouble, later ones are not tested.


\medskip


\begin{tabular}{llll}
  \\\hline
  &
  archive/date
  &
  explanation
  &
  command
  \\\hline 
  5. & \url{snapshots/slime\_v5.jar}{19.\ July 2002}
  &
  last-minute ``integration'', end-of-semester, a bit cleaned up
  &
  \texttt{java -jar slime\_v5.jar}
  \\
  4. & 18.\ July 2002
  &
  obsolete,
  &
  \\
  3. & \url{snapshots/slime\_v3.jar}{10.\ July 2002}
  &
  snapshot 3: compiles, but not integrated
  &
  \texttt{java -jar slime\_v3.jar}
  \\

  2. & \url{snapshots/slime\_v.secondcompilation-sep.jar}{9.\ July 2002}
  &
  second compilation (separately)
  &
  \texttt{java slime.editor.EditorInFrame}
  \\
  1. & \url{snapshots/slime\_v.firstcompilaton-sep.jar}{3.\ July 2002}
  &
  first compilation (separately), all packages on board
  &
  \texttt{java slime.editor.EditorInFrame}
  \\
\end{tabular}

\fi[


%%%%%%%%%%%%%%%%%%%%%%%%%%%%%%%%%%%%%%%%%%%%%%%%%%%%%%%%%%%%%%%%%%%%%%%%
%% $Id: snapshots.tex,v 1.1 2004/10/18 06:38:34 ms Exp $
%%%%%%%%%%%%%%%%%%%%%%%%%%%%%%%%%%%%%%%%%%%%%%%%%%%%%%%%%%%%%%%%%%%%%%%%
%%% Local Variables: 
%%% mode: latex
%%% TeX-master: "main"
%%% End: 




\section*{Material}

Some material (handouts, slides) which have been presented during the
course).

\begin{tabular}{r@{\quad\quad}rll}
  \hline
  1. & 19. October & coma  & 
  \href{slides/main-coma.pdf}{slides}
  \\
  2. & 19. October & cvs (obsolete) & \href{handouts/handout1.pdf}{handout 1}
  \\
  3. & 9. November & subversion + some tool remarks &
  \href{handouts/handout2.pdf}{handout 2}
  \\
\end{tabular}




\section*{Requirement specification}
\label{sec:requirement specification}

The following list contains the requirement specification
(``Pflichtenheft'').  The top-most entry is the current one

\begin{center}
  \begin{tabular}[t]{r@{\quad}l@{\quad\quad}l@{\quad\quad}p{9cm}}
    \href{requirements/v1}{Version 1}
    &
    26.\ October 2004
    & 
    original sketch of proposal, at the beginning of the semester
  \end{tabular}
\end{center}



%%%%%%%%%%%%%%%%%%%%%%%%%%%%%%%%%%%%%%%%%%%%%%%%%%%%%%%%%%
%% $Id: requirements.tex,v 1.2 2004/10/26 16:24:47 swprakt Exp $
%%%%%%%%%%%%%%%%%%%%%%%%%%%%%%%%%%%%%%%%%%%%%%%%%%%%%%%%%%


%%% Local Variables: 
%%% mode: latex
%%% TeX-master: "main"
%%% End: 


\subsection*{Weekly meetings}
\label{sec:meetings}


This section contains the results of the weekly meetings as communicated
via email to the group members. There are kept here for quick reference.




  \begin{table}[htbp]
    \centering
    \begin{tabular}[t]{r@{\quad}l@{\quad\quad}p{9cm}}
    \\\hline
    0.
    &
    \href{meetings/2004-10-19.txt}{19. October}
    &
    kick-off
    \\
    1.
    &
    \href{meetings/2004-10-26.txt}{26. October}
    &
    general jamboree
    \\
    2.
    &
    \href{meetings/2004-11-02.txt}{2. November}
    &
    Tools discussion
    \\
    3.
    &
    \href{meetings/2004-11-09.txt}{9. November}
    &
    Spec. presentations
    \\
    4.
    &
    \href{meetings/2004-11-15-a.txt}{announcement (15.11)},
    \href{meetings/2004-11-16.txt}{16. November}
    &
    group forming/next steps
    \\
    5.
    &
    \href{meetings/2004-11-23.txt}{23. November}
    &
    Various topics, in particular: status
    \\
    5.
    &
    \href{meetings/2004-11-30.txt}{30. November}
    &
    \textbf{SQL shootout}, communicaton problems,
    (\href{figures/db_photo.jpg}{blackboard} for reference)
    \\
    6.
    &
    \href{meetings/2004-12-07.txt}{7. December}
    &
    status, various topics, progress?
    \\
    7.
    &
    \href{meetings/2004-12-14.txt}{14. December}
    &
    status/progress, SQL, \textbf{testing}, svn
    \\
    8.
    &
    \href{meetings/2004-12-21.txt}{21. December}
    &
    status/progress, nothing particular
    \\\hline
    9.
    &
    \href{meetings/2005-01-11.txt}{11. January}
    &
    [status/progress], testing, DB problems
    \\
    10.
    &
    \href{meetings/2005-01-18.txt}{18. January}
    &
    status/progress report, plan for the rest
    \\
    11.
    &
    \href{meetings/2005-01-25.txt}{25. January}
    &
    nothing specific, date for end-demo
    \\
    12.
    &
    \href{meetings/2005-02-01.txt}{1. February}
    &
    last orders, handout 7
    \\
    12.
    &
    \href{meetings/2005-02-08.txt}{8. February}
    &
    ``\textbf{E-O-S}''
    \\\hline
    \iffalse
    1.
    &
    \url{meetings/meeting-160402.txt}{16. April}
    &
    task assignment, change of meeting time
    \\
    2.
    & 
    \url{meetings/meeting-240402.txt}{24. April}
    &
    no real decisions, preparation for next meeting
    \\
    3.
    &
    1.\ May
    &
    holiday
    \\
    4.
    &
    \url{meetings/meeting-080502.txt}{8.\ May}
    & 
    task presentation, status of editor group unclear,
    check group said ciao, restructuring planned
    \\
    5.
    &
    15.\ May
    &
    no email?
    \\
    6.
    &
    22. May
    &
    no email?
    \\
    7.
    &
    29.\ May
    &
    no email
    \\
    8.
    &
    \url{meetings/meeting-050602.txt}{5.\ June}
    &
    makeshift parser under utilites added; first
    integration set to 12.\ June
    \\
    9.
    &
    \url{meetings/meeting-120602.txt}{12.\ June}
    &
    restructuring now (freeze)!, makeshift checks will be implemented 
    (as visitor or otherwise), no integration (since code
    missing)
    \\
    10. 
    &
    \url{meetings/meeting-190602.txt}{19.\ June}
    &
    no integration yet
    \\
    11. 
    &
    \url{meetings/meeting-260602.txt}{26.\ June}
    &
    everyone on board; plan of final review;
    plan for last 3 weeks
    \\
    12. 
    &
    \url{meetings/meeting-030702.txt}{3.\ July}
    &
    removal of additional checked in stuff +
    removal of class files.
    \\
    13. 
    &
    \url{meetings/meeting-100702.txt}{10.\ July}
    &
    decisions about interface inconsistencies,
    making it compilable, 
    preparing the integration
    \\
    14. 
    &
    %\url{meetings/meeting-190602.txt}{17.\ July}
    17.\ July
    &
    no meeting
    \\
\fi
  \end{tabular}
    \caption{Meeting minutes}
    \label{tab:meetings}
  \end{table}



\begin{tabular}{|l|l|l|}
\hline
week & date & topic 
\\\hline
01 & 19.10. & Introduction\\
02 & 26.10. & Discussion\\
03 & 02.11. & Tools \& Architecture\\
04 & 09.11. & Specification\\
05 & 16.11. & {Milestone 0}: Specification / Tools (\LaTeX\ source)\\
06 & 23.11. &\\
07 & 30.11. &  {Milestone 1:} ...\\
08 & 07.12. &\\
09 & 14.12. &\\
10 & 21.12. & Milestone 2: ...\\
11 & 11.01. &\\
12 & 18.01. &\\
13 & 25.01. &\\
14 & 01.02. & Milestone 3: ...\\
15 & 08.02. & Presentation of final product\\
\hline
\end{tabular}



%Organisatorial/procedural things discussed during the meeting at 3.7.02
%included arguments mentioned \url{slides/meeting030702.ps}{here.}



\iffalse
Here the \important{official decision} concerning the \texttt{CLASSPATH}
etc. (same is in \texttt{Readme.devel}).

\begin{itemize}
\item the \importantxx{checked-in} versions of cup and lex are
  obligatory\footnote{They replace the ones previously used under
    \texttt{/home/java}, which had been the official ones so far. In
    effect, they are the same, and just checked in.}
\item the following \importantxx{classpath} is obligatory:
  \texttt{CLASSPATH=<WORKDIR>/Slime/src:<WORKDIR>/Slime/lib/JLex.jar:<WORKDIR>/Slime/lib/java_cup.jar:}
  where of course the \texttt{<WORKDIR>} is a placeholder and it has to be
  adapted by the individual user to his work directory.
\item no \importantxx{generated} files nor \importantxx{class} files will be
  checked in (unless technical reasons call for it, in which case we will
  discuss this in the light of the new arguments again)
\item no \importantxx{non-slime} code or data will be checked in under
  \texttt{CLASSPATH=<WORKDIR>/Slime/src}
\item revisions \emph{log}s are useful and worth reading, but it's not
  mandatory to read each other's logs.
\end{itemize}
\fi


%%% Local Variables: 
%%% mode: latex
%%% TeX-master: "main"
%%% End: 

%



%%%%%%%%%%%%%%%%%%%%%%%%%%%%%%%%%%%%%%%%%%%%%%%%%%%%%%%%%%%%
%% $Id: timeline.tex,v 1.1 2004/10/18 06:38:34 ms Exp $
%%%%%%%%%%%%%%%%%%%%%%%%%%%%%%%%%%%%%%%%%%%%%%%%%%%%%%%%%%%%


%%% Local Variables: 
%%% mode: latex
%%% TeX-master: "main"
%%% End: 

\section*{Groups}
\label{sec:groups}

%Each developer or team is responsible (in general) for one \Java-package.
%The requirements are described in the specification document.  findet sich





\begin{tabular}{|l|l|}
  \hline
  \multicolumn{2}{|c|}{Gruppe 1: ``PHP 1''}
  \\\hline\hline
  SQL &  Sandro Esquivel/Tom Scherzer
  \\\hline
  Tools & Jan Waller
  \\\hline
  Architecture &
  \begin{tabular}[t]{p{8cm}}
    Sandro Esquivel/Tom Scherzer
    \\
    Falk Starke
    \\
    Daniel Miesling
    \\
  \end{tabular}
  \\\hline
\end{tabular}
\\[4em]
\begin{tabular}{|l|l|}
  \hline
  \multicolumn{2}{|c|}{Gruppe 2 ``PHP 2''}
  \\\hline\hline
  SQL &  Gunnar Biederbeck
  \\\hline
  Tools & Torben  Dziuk
  \\\hline
  Architecture &
  \begin{tabular}[t]{p{8cm}}
    Meiko Jensen
    \\
    Marko Heiden
    \\
    Tim Fenten
    \\
    Ian Stragalis
    \\
  \end{tabular}
  \\\hline
\end{tabular}
\\[4em]
\begin{tabular}{|l|l|}
  \hline
  \multicolumn{2}{|c|}{Gruppe 3: ``Java''}
  \\\hline\hline
  SQL & Mohamed Albari
  \\\hline
  Tools & Alexander Derenbach
  \\\hline
  Architecture &
  \begin{tabular}[t]{p{8cm}}
    Oliver Wulf
    \\
    Malte Tiedje
    \\
    Peter Kauffels
    \\
    Harm Brandt
    \\
    Ulrich Schwarz
    \\
  \end{tabular}
  \\\hline
\end{tabular}
\\[4em]
\begin{tabular}{|l|p{8cm}|}
  \hline
  \multicolumn{2}{|c|}{Test}
  \\\hline\hline
   & 
   Thiago Tonelli Bartolomei
   \\
   &Olle Nebendahl
   \\
   &
   Oliver Niemann
  \\\hline
\end{tabular}




\subsection*{Phase 1}

In the first phase, we do not have the final groups, it's a handful of
``task forces'' to prepare the tools, the spec, and a testing concept.

\begin{verbatim}

Spec1:

    Tom Scherzer
    Sandro Esquivel
    Jan Waller 

Spec2:

    Ulrich Schwarz 
    Falk Starke 
    Daniel Miesling

Test:
     Oliver Niemann 
     Tonelli Bartolomei Thiago

Tools1:

     Marco Heyden 
     Gunnar Biederbeck
     Ioannis Stragalis 
     Mohamed Ziad Albari

Tools2:

     Adriana Lukaschewitz
     Malte Tiedje 
     Harm Brandt
     Peter Kauffels 

 Tool3

     Meiko Jensen
     Thorben Dziuh 
     Oliver Wulf 
     Tim Fenten


\end{verbatim}

\iffalse



\begin{table}[htbp]
  \centering
  \begin{tabular}[t]{l@{\quad\quad}l}
     package  &  responsible
     \\\hline
     gui/integration & Norbert Boeck
     \\
     editor & Andreas Niemann
     \\
     checks & 
     Karsten Stahl,
     \url{http://www.informatik.uni-kiel.de/\home{ms}}{Martin Steffen}
     \\
     simulator & Immo Grabe
     \\
     parser & Marco Wendel,
     \\
     (layout) & Andreas Niemann
     \\
     abstract syntax & all
     \\
     utils 
     &
     Karsten Stahl,
     \url{http://www.informatik.uni-kiel.de/\home{ms}}{Martin Steffen}
  \end{tabular}
  \caption{}
  \label{tab:gruppen}
\end{table}




The \emph{email adresses} of the developers are available (internally) at
\begin{verbatim}
      $WORKDIR/Slime/org/packages.txt
\end{verbatim}
For messages to the whole project, use
%\mathtt{swprakt+slime@informatik.uni-kiel.de}$.
\fi


%%% Local Variables: 
%%% mode: latex
%%% TeX-master: "main"
%%% End: 

%


\section*{Conventions and rules of the game}
\label{sec:conventions}



Besides the \emph{CVS-strategies} which have been handed out and discussed,
we should take care of the following:

\begin{itemize}
\item \textbf{Makefile}: each package, i.e., the root of the respective
  sub-directory must containt a \texttt{Makefile}.  Its first target must
  be \texttt{make all}, which generates for the packges the \Java{} byte
  code (whatever it takes in intermediate steps).  Additionally, a target
  \texttt{make clean} must be supported, which removes the byte-code and
  all temporary files again.  An example for a workable Makefile is
  contained in the \texttt{absynt}-package which covers the (\emph{very
    simple}) needs just described.

  
\item \textbf{Documentation:} the \Java-code should be meaningfully
  commented and documented. Information about the implementations which are
  relevant for the co-developers, are done in \javadoc. This especially
  applies to the intended meaning and usage of the public interfaces
  between the packages. Part of the documentation should be the name of the
  developer(s) and the version of the file. Further useful information
  could be the status of the method, class, \ldots, e.g., whether the
  implementation currently is only available as stub, whether the
  functionality is finished, but not yet tested, which are the known
  bugs/features \ldots.
  
  The \javadoc-comments may serve the internal communication to state what
  the package offers (and what it assumes). The generated web-pages are
  available via the project page, they are refreshed from time to time, and
  kept
  \url{http://www.informatik.uni-kiel.de/inf/deRoever/SS02/Java/Slime/doc}{here}.
\end{itemize}



%%%%%%%%%%%%%%%%%%%%%%%%%%%%%%%%%%%%%%%%%%%%%%%%%%%%%%%%%%%%
%% $Id: conventions.tex,v 1.1 2004/10/18 06:38:33 ms Exp $
%%%%%%%%%%%%%%%%%%%%%%%%%%%%%%%%%%%%%%%%%%%%%%%%%%%%%%%%%%%%
%%% Local Variables: 
%%% mode: latex
%%% TeX-master: "main"
%%% End: 





\section{Links}


Thanks to Alexander for the collection of Links

\begin{itemize}
\item \href{http://www.apache.org}{Apache}
\item \href{http://www.mysql.org}{Mysql}
\item \href{http://java.sun.com/j2se/index.jsp}{J2SE}
\item \href{http://java.sun.com/j2ee/index.jsp}{J2EE}
\item \href{http://java.sun.com/products/ejb/}{EJB}
\item \href{http://java.sun.com/products/jsp/index.jsp}{JSP}
\item \href{http://java.sun.com/products/servlet/index.jsp}{Servelets}
\item \href{http://www.mysql.com/products/connector/j/}{JDBC}
\item \href{http://jakarta.apache.org/tomcat/index.html}{Jacarta Tomcat}
\item \href{http://www.jboss.org/}{JBoss App. Server}
\end{itemize}



\bibliographystyle{alpha}



\bibliography{string,etc,oop,crossref}


\end{document}


%%% Local Variables: 
%%% mode: latex
%%% TeX-master: t
%%% End: 


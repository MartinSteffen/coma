


\newif\ifweb\webtrue
\def\version{5}
\def\status{transient}
\def\docdate{17.~01.~2005}




%%%%%%%%%%%%%%%%%%%%%%%%%%%%%%%%%%%%%%%%%%%%%%%%%%%%%%%%%%%%%%%
%% $Id$ %%
%%%%%%%%%%%%%%%%%%%%%%%%%%%%%%%%%%%%%%%%%%%%%%%%%%%%%%%%%%%%%%%
%%% Local Variables: 
%%% mode: latex
%%% TeX-master: "main"
%%% End: 


\documentclass[11pt]{article}

\usepackage{hevea}
\usepackage{hyperref}
%\usepackage{url}
\usepackage[english]{babel}
\usepackage{epsfig}
\usepackage{tfheader}
\usepackage{a4wide}
\usepackage[latin1]{inputenc}

\usepackage{graphicx}
\graphicspath{{../figures/}} %% cool!

\usepackage{psfrag}

\ifweb

\htmlfoot{\hrulefill{}
  {\footnotesize Pages last (re-)generated \today}}
\renewcommand{\@bodyargs}{bgcolor="white" alink="red" vlink="\#407999"  link="\#7070ff"}  
\fi







%%% Local Variables: 
%%% mode: latex
%%% TeX-master: "main"
%%% End: 

%% common macro files for the projects


\newcommand{\NONAME}{Somename}


\newcommand{\Coma}{\textsl{Coma}}

\newcommand{\cvs}{cvs}
\newcommand{\Java}{\textsc{Java}}
\newcommand{\Cplusplus}{C$^{++}$}
\newcommand{\javadoc}{\textsc{javadoc}}



\newcommand{\team}[1]{\textbf{Responsible:} #1\bigskip{}}


\newcommand{\bnfdef}               {::=}
\newcommand{\bnfbar}               {\mathrel{|}}

\newcommand{\suchthat}{\mathrel{\mid}}
\newcommand{\union}  {\cup}
\newcommand{\intersect}  {\cap}
\newcommand{\sizeof}[1]  {{\mid}{#1}{\mid}}
\newcommand{\sem}    [2]        {[\![#2]\!]_{#1}}      %semantics

\newenvironment{diagram}{\begin{displaymath}}{\end{displaymath}}


%%% Local Variables: 
%%% mode: latex
%%% TeX-master: t
%%% End: 




\title{{\huge\bf Coma} (under constant construction)}
\author{WS0405}
%\url{}{Martin Steffen}}
\date{}




\begin{document}
\vspace{-2cm}


%\pagestyle{empty}

%%%% common header decoration (used for TeX-generated web-pages)


\begin{rawhtml}

<a href="http://www.uni-kiel.de/">
  <img border=0 alt="[Christian-Albrechts-Universit�t]" height=30  src="http://www.uni-kiel.de/img/cau_logo.gif" align="left"></a>


<a href="http://www.tf.uni-kiel.de/">
  <img border=0 alt="[Technical Faculty]" height=30  src="/images/tflogo.gif" align="right"></a>
<br>

\end{rawhtml}



%%% Local Variables: 
%%% mode: latex
%%% TeX-master: t
%%% End: 


\maketitle{}



\begin{abstract}
  This is the developer's main page of the ``Conference Manager''
  (\Coma)-project of the winter term 2004/05. It collects most of the
  top-level information during the development process.

  \begin{center}
    \textbf{Version:} \texttt{$Id$}
  \end{center}

  \medskip
  
  \textbf{Disclaimer:} The material here -the webpages, the documentation,
  in particular the \emph{code} \ldots- is maintained by the participants
  of the course, i.e., the students + the organizers. It is collected and
  made available for teaching purposes, in particular for non-profit
  purposes. Neither the university nor the organizers of the course offer
  any guarantee whatsoever concerning correctness, absence of errors,
  fitness for the purpose announced or a purpose assumed etc. In a word:
  \emph{have fun, learn, download if you wish, play around, but don't
    complain.}
\end{abstract}

%\tableofcontents



%
\section*{Review}
\label{sec:review}

There will be a final review, demo, presentation, and debriefing at the end of
the semester.


\iffalse
Date of the final review, demo, and debriefing is fixed to
\begin{center}
  \importantxx{\textbf{18.07.2002, 17:00}} 
\end{center}



\begin{table}[htbp]
  \centering
  \begin{tabular}{ll}
    Karsten Stahl, Martin Steffen & \url{review/overview.pdf}{overview}
    \\
    Andreas Niemann & \url{review/editor.pdf}{editor}
    \\
    Marco Wendel & \url{review/parser.pdf}{parser}
    \\
    Immo Grabe & simulator
    \\
    Karsten Stahl, Martin Steffen & \url{review/checks.pdf}{checks}
  \end{tabular}
  \caption{Presentations}
  \label{tab:presentations}
\end{table}

\fi



%%%%%%%%%%%%%%%%%%%%%%%%%%%%%%%%%%%%%%%%%%%%%%%%%%%%%%
%% $Id: review.tex,v 1.1 2004/10/18 06:38:33 ms Exp $
%%%%%%%%%%%%%%%%%%%%%%%%%%%%%%%%%%%%%%%%%%%%%%%%%%%%%%
%%% Local Variables: 
%%% mode: latex
%%% TeX-master: "main"
%%% End: 



%\section*{Java-doc}
%\label{sec:javadoc}
%\label{sec:html-doc}
%For convenient lookup for those who like \javadoc, the public documentation
%and comments are avaiable here. The pages are refreshed from time to time,
%especially after larger development steps.
%\begin{center}
%  \importantxx{documentation \url{doc}{Slime}}: currently (\today): last-minute
%  ``integrated'' tool
%\end{center}

%\section*{Snapshots}
\label{sec:snapshots}


The \emph{baselines} or \emph{snapshots} are archived here for quick
reference. %


\iffalse as \texttt{slime\_v[x].jar}, where \texttt{[x]} is the number or the
tag of the snapshot. Developers with access to the source repository can
retrieve those snapshots of course with cvs.\footnote{The command is
  \texttt{cvs update -r [tag]}. Be careful with this command, don't generate
  development branches without knowing.}


To \emph{execute} one of the archived snapshots, save it at some convenient
place, set the \texttt{CLASSPATH} (or use the \texttt{-classpath} option)
of \texttt{java}) to include the saved jar-file + the required versions of
\texttt{JLex} and \texttt{java\_cup}, and do\footnote{Since the archive is
  not stand-alone in that it requires additionally at run-time the lexer
  and the parser generator, the archive is not executable with the
  \texttt{-jar} option.}

\begin{center}
  \texttt{java <slimeversion>.jar slime.Main}
\end{center}


The \emph{required} version is \texttt{jdk1.4,} earlier version may cause
trouble, later ones are not tested.


\medskip


\begin{tabular}{llll}
  \\\hline
  &
  archive/date
  &
  explanation
  &
  command
  \\\hline 
  5. & \url{snapshots/slime\_v5.jar}{19.\ July 2002}
  &
  last-minute ``integration'', end-of-semester, a bit cleaned up
  &
  \texttt{java -jar slime\_v5.jar}
  \\
  4. & 18.\ July 2002
  &
  obsolete,
  &
  \\
  3. & \url{snapshots/slime\_v3.jar}{10.\ July 2002}
  &
  snapshot 3: compiles, but not integrated
  &
  \texttt{java -jar slime\_v3.jar}
  \\

  2. & \url{snapshots/slime\_v.secondcompilation-sep.jar}{9.\ July 2002}
  &
  second compilation (separately)
  &
  \texttt{java slime.editor.EditorInFrame}
  \\
  1. & \url{snapshots/slime\_v.firstcompilaton-sep.jar}{3.\ July 2002}
  &
  first compilation (separately), all packages on board
  &
  \texttt{java slime.editor.EditorInFrame}
  \\
\end{tabular}

\fi[


%%%%%%%%%%%%%%%%%%%%%%%%%%%%%%%%%%%%%%%%%%%%%%%%%%%%%%%%%%%%%%%%%%%%%%%%
%% $Id: snapshots.tex,v 1.1 2004/10/18 06:38:34 ms Exp $
%%%%%%%%%%%%%%%%%%%%%%%%%%%%%%%%%%%%%%%%%%%%%%%%%%%%%%%%%%%%%%%%%%%%%%%%
%%% Local Variables: 
%%% mode: latex
%%% TeX-master: "main"
%%% End: 




{\centering
  \begin{minipage}{10cm}
    \begin{tabular}[t]{l}
    \em
    Ja, mach nur einen Plan, 
    \\
    \emph{sei nur ein kluges Licht, }
    \\ 
    \emph{und mach dann noch 'nen zweiten Plan, }
    \\
    \emph{geh'n tun sie beide nicht.}
    \\
    \textbf{\em Bert Brecht, Die Dreigroschenoper}
\end{tabular}
  \end{minipage}}




\section*{Requirement specification}
\label{sec:requirement specification}

The following list contains the requirement specification
(``Pflichtenheft'').  The top-most entry is the current one

\begin{center}
  \begin{tabular}[t]{r@{\quad}l@{\quad\quad}l@{\quad\quad}p{9cm}}
    \href{requirements/v1}{Version 1}
    &
    26.\ October 2004
    & 
    original sketch of proposal, at the beginning of the semester
  \end{tabular}
\end{center}



%%%%%%%%%%%%%%%%%%%%%%%%%%%%%%%%%%%%%%%%%%%%%%%%%%%%%%%%%%
%% $Id: requirements.tex,v 1.2 2004/10/26 16:24:47 swprakt Exp $
%%%%%%%%%%%%%%%%%%%%%%%%%%%%%%%%%%%%%%%%%%%%%%%%%%%%%%%%%%


%%% Local Variables: 
%%% mode: latex
%%% TeX-master: "main"
%%% End: 



\section*{Material}

Some material (handouts, slides) which have been presented during the
course).

\begin{tabular}{r@{\quad\quad}rll}
  \hline
  1. & 19. October & coma  & 
  \href{slides/main-coma.pdf}{slides}
  \\
  2. & 19. October & cvs (obsolete) & \href{handouts/handout1.pdf}{handout 1}
  \\
  3. & 9. November & subversion + some tool remarks &
  \href{handouts/handout2.pdf}{handout 2}
  \\
   & 16. November & bugzilla &
  \href{handouts/handout3.pdf}{handout 3}
  \\
   & 15. November & groups &
  \href{handouts/handout4.pdf}{organizational form}
  \\
   & 15. November & groups and next steps &
  \href{handouts/handout5.pdf}{handout 5}
  \\\hline 
   & 18. November &  &
  \href{spec/v2/main.pdf}{data spec} (v2)
  \\\hline
\end{tabular}




\subsection{Spec + tools groups}
\label{sec:material.spec-tools}


\begin{tabular}{llll}
  14.11 & Spec 1 & 
  \href{spec/material/spec1/spezifikation.pdf}{specification}, 
  \href{spec/material/spec1/slides.pdf}{coma slides}, 
  & (in German)
  \\
  16.11 & Spec 2 & 
  \href{spec/material/spec2/usecase.pdf}{usecase}, 
  \href{spec/material/spec2/datenstruktur.ps}{Datenstruktur}, 
  \href{spec/material/spec2/slides.pdf}{coma slides}
  & (in German)
  \\
  10.11 & Alexander & \href{spec/material/derenbach/tools.pdf}{choice of tools}
  \\
  ?  & Tools 1 &
  \\
  15.11.04 & Tools 2 &
  \href{spec/material/tool2/tools.pdf}{choice of tools}
  \\
  16.11 & Tools-3 & \href{spec/material/tools3/tools.pdf}{choice of tools}
\end{tabular}




\subsubsection*{Other stuff}

\begin{tabular}{lp{8cm}}
 \LaTeX\ \href{misc/handout.cls}{handout class}  +
 \href{misc/hevea.sty}{hevea} & for tex'ing the handouts, if wished.
 \\
 \href{http://snert.informatik.uni-kiel.de:8080/~swprakt/phpBB2/}{Bulletin Boad}  
 \\
 \href{http://snert.informatik.uni-kiel.de:8080/~swprakt/bugzilla}{bugzilla}  
 &
 login via email-address + your chosen password (not the generated one, not the ssh-passphrase)
\end{tabular}





%%% Local Variables: 
%%% mode: latex
%%% TeX-master: "main"
%%% End: 


%



%%%%%%%%%%%%%%%%%%%%%%%%%%%%%%%%%%%%%%%%%%%%%%%%%%%%%%%%%%%%
%% $Id: timeline.tex,v 1.1 2004/10/18 06:38:34 ms Exp $
%%%%%%%%%%%%%%%%%%%%%%%%%%%%%%%%%%%%%%%%%%%%%%%%%%%%%%%%%%%%


%%% Local Variables: 
%%% mode: latex
%%% TeX-master: "main"
%%% End: 

\section*{Groups}
\label{sec:groups}

%Each developer or team is responsible (in general) for one \Java-package.
%The requirements are described in the specification document.  findet sich





\begin{tabular}{|l|l|}
  \hline
  \multicolumn{2}{|c|}{Gruppe 1: ``PHP 1''}
  \\\hline\hline
  SQL &  Sandro Esquivel/Tom Scherzer
  \\\hline
  Tools & Jan Waller
  \\\hline
  Architecture &
  \begin{tabular}[t]{p{8cm}}
    Sandro Esquivel/Tom Scherzer
    \\
    Falk Starke
    \\
    Daniel Miesling
    \\
  \end{tabular}
  \\\hline
\end{tabular}
\\[4em]
\begin{tabular}{|l|l|}
  \hline
  \multicolumn{2}{|c|}{Gruppe 2 ``PHP 2''}
  \\\hline\hline
  SQL &  Gunnar Biederbeck
  \\\hline
  Tools & Torben  Dziuk
  \\\hline
  Architecture &
  \begin{tabular}[t]{p{8cm}}
    Meiko Jensen
    \\
    Marko Heiden
    \\
    Tim Fenten
    \\
    Ian Stragalis
    \\
  \end{tabular}
  \\\hline
\end{tabular}
\\[4em]
\begin{tabular}{|l|l|}
  \hline
  \multicolumn{2}{|c|}{Gruppe 3: ``Java''}
  \\\hline\hline
  SQL & Mohamed Albari
  \\\hline
  Tools & Alexander Derenbach
  \\\hline
  Architecture &
  \begin{tabular}[t]{p{8cm}}
    Oliver Wulf
    \\
    Malte Tiedje
    \\
    Peter Kauffels
    \\
    Harm Brandt
    \\
    Ulrich Schwarz
    \\
  \end{tabular}
  \\\hline
\end{tabular}
\\[4em]
\begin{tabular}{|l|p{8cm}|}
  \hline
  \multicolumn{2}{|c|}{Test}
  \\\hline\hline
   & 
   Thiago Tonelli Bartolomei
   \\
   &Olle Nebendahl
   \\
   &
   Oliver Niemann
  \\\hline
\end{tabular}




\subsection*{Phase 1}

In the first phase, we do not have the final groups, it's a handful of
``task forces'' to prepare the tools, the spec, and a testing concept.

\begin{verbatim}

Spec1:

    Tom Scherzer
    Sandro Esquivel
    Jan Waller 

Spec2:

    Ulrich Schwarz 
    Falk Starke 
    Daniel Miesling

Test:
     Oliver Niemann 
     Tonelli Bartolomei Thiago

Tools1:

     Marco Heyden 
     Gunnar Biederbeck
     Ioannis Stragalis 
     Mohamed Ziad Albari

Tools2:

     Adriana Lukaschewitz
     Malte Tiedje 
     Harm Brandt
     Peter Kauffels 

 Tool3

     Meiko Jensen
     Thorben Dziuh 
     Oliver Wulf 
     Tim Fenten


\end{verbatim}

\iffalse



\begin{table}[htbp]
  \centering
  \begin{tabular}[t]{l@{\quad\quad}l}
     package  &  responsible
     \\\hline
     gui/integration & Norbert Boeck
     \\
     editor & Andreas Niemann
     \\
     checks & 
     Karsten Stahl,
     \url{http://www.informatik.uni-kiel.de/\home{ms}}{Martin Steffen}
     \\
     simulator & Immo Grabe
     \\
     parser & Marco Wendel,
     \\
     (layout) & Andreas Niemann
     \\
     abstract syntax & all
     \\
     utils 
     &
     Karsten Stahl,
     \url{http://www.informatik.uni-kiel.de/\home{ms}}{Martin Steffen}
  \end{tabular}
  \caption{}
  \label{tab:gruppen}
\end{table}




The \emph{email adresses} of the developers are available (internally) at
\begin{verbatim}
      $WORKDIR/Slime/org/packages.txt
\end{verbatim}
For messages to the whole project, use
%\mathtt{swprakt+slime@informatik.uni-kiel.de}$.
\fi


%%% Local Variables: 
%%% mode: latex
%%% TeX-master: "main"
%%% End: 

%


\section*{Conventions and rules of the game}
\label{sec:conventions}



Besides the \emph{CVS-strategies} which have been handed out and discussed,
we should take care of the following:

\begin{itemize}
\item \textbf{Makefile}: each package, i.e., the root of the respective
  sub-directory must containt a \texttt{Makefile}.  Its first target must
  be \texttt{make all}, which generates for the packges the \Java{} byte
  code (whatever it takes in intermediate steps).  Additionally, a target
  \texttt{make clean} must be supported, which removes the byte-code and
  all temporary files again.  An example for a workable Makefile is
  contained in the \texttt{absynt}-package which covers the (\emph{very
    simple}) needs just described.

  
\item \textbf{Documentation:} the \Java-code should be meaningfully
  commented and documented. Information about the implementations which are
  relevant for the co-developers, are done in \javadoc. This especially
  applies to the intended meaning and usage of the public interfaces
  between the packages. Part of the documentation should be the name of the
  developer(s) and the version of the file. Further useful information
  could be the status of the method, class, \ldots, e.g., whether the
  implementation currently is only available as stub, whether the
  functionality is finished, but not yet tested, which are the known
  bugs/features \ldots.
  
  The \javadoc-comments may serve the internal communication to state what
  the package offers (and what it assumes). The generated web-pages are
  available via the project page, they are refreshed from time to time, and
  kept
  \url{http://www.informatik.uni-kiel.de/inf/deRoever/SS02/Java/Slime/doc}{here}.
\end{itemize}



%%%%%%%%%%%%%%%%%%%%%%%%%%%%%%%%%%%%%%%%%%%%%%%%%%%%%%%%%%%%
%% $Id: conventions.tex,v 1.1 2004/10/18 06:38:33 ms Exp $
%%%%%%%%%%%%%%%%%%%%%%%%%%%%%%%%%%%%%%%%%%%%%%%%%%%%%%%%%%%%
%%% Local Variables: 
%%% mode: latex
%%% TeX-master: "main"
%%% End: 









%\bibliographystyle{alpha}
%\bibliography{string,etc,oop,crossref}

\end{document}


%%% Local Variables: 
%%% mode: latex
%%% TeX-master: t
%%% End: 




\documentclass{article}
\usepackage[german]{babel}
\usepackage[latin1]{inputenc}
\usepackage{amsmath,amssymb}

\begin{document}

\section*{Status PHP2, 18.01.2005}
\subsection*{Grobe Modul"ubersicht und ''Stand der Dinge'':}
\begin{description}
\item[Admin \& Install] Die Implementierung dieses Moduls ist
  abgeschlossen, d.h. der gew"unschte Funktionsumfang ist vorhanden.  Es
  besteht eine Spec-Beschreibung, die den Zweck einer README erf"ullt.  Das
  Modul wird zur Zeit getestet.
\item[Chair] Das Chairmodul ist ebenfalls komplett implementiert und
  "ubernimmt zus"atzlich einige Funktionen, die nicht direkt in seinen
  Bereich fallen, zB die Benutzerer-Registrierung und die Rollenzuweisung
  an neue Teilnehmer einer Konferenz.  Es besteht eine Spec-Beschreibung,
  die die Funktionalit"aten erkl"art.  Das Modul wir zur Zeit getestet.
\item[Reviewer] Das Modul Reviewer ist nahezu komplett spezifiziert, was
  die Implementierung erleichtert. Diese \emph{Implementierung} steht
  allerdings noch aus.  Die im spec-Verzeichnis aufgef"uhrten Eintr"age,
  geben einen "Uberblick "uber das Modul.
\item[Author] Das Modul Author ist zum Teil implementiert, es fehlt jedoch
  mit der Datei-Verwaltung via FTP noch das Kernst"uck. Dieses wird in
  absehbarer Zeit hinzugef"ugt. (Siehe dazu auch unten) Es besteht bisher
  keine schriftliche Spezifikation.
\item[Forum] Das Forum ist spezifiziert, die Implementierung ist jedoch
  hintenangestellt, bis die kritischen Bereiche abgeschlossen sind.
\item[Algorithmen] Die zwei wesentlichen Algorithmen,
  (PaperToReviewerAlgorithm, RatingToProgramAlgorithm) sind spezifiziert
  und werden zur Zeit implementiert.
\item[Struktur, Rechte, Rollen] Das Grundsystem des Tools ist lauff"ahig
  und erprobt, es ist essentiell f"ur die Modulverwaltung zur Laufzeit. Aus
  diesem Bereich sind keine Fehler zu erwarten.  Das Rechtesystem l"auft
  zur Zeit stumm mit und wird durch Sicherheitsabfragen aufgefangen, es
  kann jedoch ohne grossen Aufwand hinzugenommen werden.
\item[Templates \& Design] Eine auf dem Grundger"ust aufsetzende
  Templateverwaltung regelt die Ausgabe des Tools. Diese funktioniert
  bisher fehlerlos. Auch aus dieser Richtung sind keine weiteren Fehler zu
  erwarten.  Die Templates selber sind Teil jedes Moduls und werden vom
  jeweiligen Verantwortlichen selbst erstellt. Somit ist dieses Modul erst
  mit allen anderen Modulen zusammen abgeschlossen, es fehlen hier jedoch
  keine kritischen Bereiche mehr.
\end{description}

\subsection*{Deadline 31.01.2005}
Es besteht eine "Ubereinkunft innerhalb der Gruppe, dass das Tool bis zum
31.01.2005 einschliesslich fertiggestellt werden muss, um den Testern
wenigstens einen minimalen Zeitraum mit vollem Funktionsumfang zu geben.

Da noch in einigen Bereichen wesentliche, kritische Teile fehlen, ist
dieser Termin verpflichtend f"ur alle. Bis dahin m"ussen fertiggestellt
sein:\ 
\begin{quote}
  Dateiverwaltung des Authormoduls, Bewertungsfunktionen des Reviewermoduls
\end{quote} 
Ausserdem sollen fertiggestellt sein:
\begin{quote}
  PaperToReviewerAlgorithm, RatingToProgramAlgorithm
\end{quote}
  
  Als ''Plan B'' wird lediglich wie bereits erw"ahnt, das Forum als
  entbehrlich angesehen, da dieses auch zB durch die Eingliederung eines
  PHPBB ersetzt werden kann. Die Algorithmen k"onnen (aber sollen nicht)
  durch zB einfachste Randomverteilungen ersetzt werden, dies sit jedoch
  nicht w"unschenswert.
  
\end{document}


%%% Local Variables: 
%%% mode: latex
%%% TeX-master: t
%%% End: 

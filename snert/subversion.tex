\documentclass[a4paper,11pt]{article}

\begin{document}

\begin{abstract}
  We document the configuration of subversion on snert.
\end{abstract}


\section{Configuration Files}
\label{sec:config}

The configuration of the subversion repository can be found in
\texttt{/etc/httpd/conf.d/repository.conf}.  The apache web server
will read this file when it is started.

The configuration of the user access rights to the repository for the
COMA projects can be found in
\texttt{\~{}swprakt/public\_html/svnaccess}, where the user
names are defined in
\texttt{\~{}swprakt/public\_html/svnpasswd}.

The \texttt{svnpasswd} file has to be modified only by using the
\texttt{htpasswd} tool.  We only use MD5 encryption for passwords, so
you always have to pass the command line option \texttt{-m} to
\texttt{htpasswd}.



\section{Security}
\label{sec:security}

It is necessary that the repository is only accessed by the user
apache.  To enable administration of the repository, the account
\texttt{swprakt} is allowed to switch is user to \texttt{apache} and
execute subversion related commands (\texttt{svn}, \texttt{svnlook},
and \texttt{svnadmin}).  Generally, issuing a \texttt{sudo -u apache
  svnadmin recover \~{}/public\_html/svnroot} will get the repository
running again, if something fails.

For the same reason it is not possible or advised to access the
repository using \texttt{file://} URI's to access the repository.
Only use \texttt{http://snert:8080/svn/coma} as the root of the
repository.

\end{document}
